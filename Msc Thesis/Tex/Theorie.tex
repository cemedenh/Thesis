% !TeX spellcheck = en_GB
% !TeX spellcheck = en_US 
\chapter{Theoretical Background}

In the following chapter we will present all the theoretical background necessary for the development of this project, from the theory and creation of the He and Ne droplets to the physics theories of plasma and cluster coulomb explosion processes including its detection techniques. In order to guide the reader in an organised way, the chapters are organized in a way that follow the processes necessaries to the performance of the experiment. 


\section{Helium Nanodroplets}

The combination of cryogenic matrix isolation, discovered in 1954 \cite{whittle_matrix_1954}, and the now well defined properties of Helium ($He$), specially its superfluidity face discovered in 1937 by \textit{Kapitza et. all}            \cite{kapitza_viscosity_1938}, have as consequence one of the most powerful and flexible tool in physics, the helium nanodroplets.
Helium nanodrops  have unique properties that makes it a unique source for the cluster and nanophysics experiments in the last decades. For example, they do not exhibit any optical transitions in the entire infrared, visible and ultraviolet range. They can readily pick up atoms and molecules and  form different complexes of the species embedded in their interior or on their surfaces, acting as a ideal matrix for atoms, molecules and clusters isolation. \cite{stienkemeier_spectroscopy_2006}\cite{toennies_superfluid_2004}.
The size of a He cluster can go from of a few thousands up to $10^{8}$ of atoms, and reach the ultra cold temperature regime (close to 0.37$K$ \cite{toennies_spectroscopy_1998})\cite{enss_low-temperature_2005}.
Two main advantages of this  cooling properties arise. First,  dopants in the He nanodroplet are set to their absolute vibronic ground states, avoiding all other possible espectra and stablishing the cluster in a specific state, more important, the fast cooling helps to the formation of isomers that are difficult or impossible to generate with other methods \cite{nauta_nonequilibrium_1999}. Second, because the superfluid fase of the He nanodroplets\cite{grebenev_superfluidity_1998}, the bond between dopants and He is weak. Therefore, in contrast to spectroscopy in other matrices with higher temperatures, the optical transitions of many dopants are barely influenced by the He matrix \cite{toennies_superfluid_2004}. 
The theory of  He superfluidity will not be part of this section, this imformation is well documented in other sources, and here we are based on ref.\cite{enss_low-temperature_2005} where all theory is well presented to the reader. In the next section we will dedicate a bigger effort to explain the theoretical and technical background of the He nanodroplets creation as well as the physical and technical process to doped it. 

\subsection{He Nano droplets production}

At room temperature, helium is a light inert gas. It is odorless, colorless, tasteless, and after hydrogen, the second most abundant element in the universe.  \cite{enss_low-temperature_2005}. It have a simple 2 atoms structure, exhibing numerous exotic phenomena whose theoretical descriptions are rather complex in many cases, i.e it characteristics of  a quantum fluid. From helium exist  two stable isotopes $^{3}He$ and $^{4}He$.  $^{4}He$ has two electrons, two protons and two neutrons, no nuclear spin and no total spin, pertaining to the bosonic family, while $^{3}He$ with only one neutron has a spin of $I = 1/2$ and belongs to the fermions \cite{atkins_liquid_2014}.

The bosonic state $_{4}He$ is specially of interest, at  temperature T$\leqslant$2.8K and under normal pressure has a phase transition from "normal liquid" He-I to super liquid He-II \cite{swenson_liquid-solid_1950}, in which the helium can be described by a Bose-Einstein condensation. Even the fermionic $^{3}He$ exhibits this phase transition at T$\leqslant 0.03K$ \cite{halperin_properties_1978}.

The superfluidity of $He-II$, at temperatures close to absolute zero, brings with it some unique features. The essential Properties for this include an almost disappearing viscosity in the superfluid phase, weak interaction, very efficient cooling, and the Transparency for electromagnetic radiation up to wavelengths in vacuum ultraviolets (VUV) Spectral range \cite{enss_low-temperature_2005}. Helium has therefore in the complete visible spectrum no transitions from the ground state. Through the noble gas configuration, helium has a spherically symmetrical electron distribution \cite{lewis_helium_2014}, it can hardly be polarized and is the least reactive of all the elements.

\begin{figure}[hbtp]
\caption{$^{4}He$ Phase transition at Ultra cold temperatures. $^{4}He$ is the more common isotope of helium. It remains liquid at zero temperature if the pressure is below $2.5 MPa$. The liquid has a phase transition to a superfluid phase, also known as He-II, at the temperature of $2.17 K$ (at vapor pressure). Taken from \cite{noauthor_helium_nodate}}
\centering
\includegraphics[scale=1]{../Images/He_temp_phases.png}
\end{figure}


\subsubsection{Helium Droplets}

The production He droplets had to overcome first one principal problem, its liquefaction. At the end  of 19th century many gases were liquefied for the first time by applying pressure at room temperature. However, for He and hydrogen, this method was not successful. In 1922 Kamerlingh Onnes reached temperatures below $1K$ by reducing the vapor pressure above liquid helium to about $2*10^{-5}$ bar with a series of pumps \cite{van_delft_discovery_2010}. The Joule–Thomson effect \cite{weinberger_discovery_2013} is in this case the responsible for Onnes experiment to reach this low temperatures. The basic idea is that under suitable conditions a gas in expanding performs work against its internal forces. Basically the gas is expanded through a small nozzle thermally isolated from its surroundings. The expansion under theses conditions takes place at constant enthalpy, since the expansion nozzle performs no work. following the next relation:

\begin{equation}
W= H_{1}-H_{2} = (U_{1}+p_{1}V_{1})-(U_{2}+p_{2}V_{2})
\end{equation}

where H is the entalpy before and after, $U=\dfrac{3}{2}Nk_{b}T$ for ideal gases and $pV=Nk_{b}T$ \cite{enss_low-temperature_2005}. Under Joule–Thomson effect conditions, $W=0$ so $H_{1}=H_{2}$, this expansion leads to a cooling or a warming and under certain conditions, becomes supersaturated. As a result, condensation takes place and a beam of clusters is formed.

Helium nanodroplets are typically produced by a continuous or pulsed adiabatic Expansion of pre$-$cooled helium through a small aperture from a reservoir into a vacuum  \cite{stienkemeier_spectroscopy_2006}. In this process a droplet jet is formed, and its characteristics (blasting speeds and size distribution) can be changes due the manipulation of the set$-$up. For example, $\bigtriangleup$ pressure between the reservoir and the vacuum chamber (usually in the range of a few to $10MPa$) , the nozzle temperatures(from a few K to $T \leqslant 40K$) or the nozzle size (with pinholes of diameter rounding  $5-20 \mu m$).



\begin{figure}[h!]
\centering
	\includegraphics[width=0.6\textwidth]{../Images/jet_scketch.png}
	\caption[Scheme for a nozzle expansion]{ a) Schematic representation of the processes leading to the formation and subsequent cooling of helium droplets in a gas expansion. b) Calculated dependence of the droplet temperature on time for $^{4}He$ and $^{3}He$ droplets after they have left the cluster, taken from \cite{toennies_superfluid_2004}	}
	\label{img:jet}	
\end{figure}

\begin{figure}[h!]
\centering
	\includegraphics[width=0.5\textwidth]{../Images/waanderwaal_hehe.PNG}
	\caption[Waan der Wall He-He potential]{ Waan der Wals potential for He-He interaction}
	\label{img:WanderHe}
\end{figure}


When the Helium expands from the nozzel, its thermal energy is transform in kinetic energy of a supersonic flow field. After the expantion into the vacuum, the gas becomes supersaturated and condensations starts to occurs, creating the beam clusters. This clusters are made of atoms or moleulces held togueter by Wannder wals fores, in this case He-He interaction, that share the same kinetic vector. This means that the two particules travel as close and parallel to each other that a bonding is possible, see fig \ref{img:WanderHe}. From the reference frame of the cluster, each of its molecules are close to cero movments, in He this enhace the conditions to be liquid and in consecience superfluidity is achive  \cite{hagena_cluster_1972}.
 
Depending on the buffer gas used, the mechanisms for cluster formation in the supersonic expansion range from condensation from the gas phase to fragmentation of a liquid phase. figure \ref{img:t-s_He} show how to distinguish between these two limiting conditions. In the case the expansion is isentropic (adiabatic and reversible), the expansion is represented by a vertical line in this diagram. Clusters formed by condensation from the gas phase occur when the expansion crosses into the two-phase region on the right-hand side of the critical point. Clusters formed by fragmentation of the liquid phase occur when the expansion crosses into the two-phase region on the left hand side of the critical point. the diagram is an example of three gases, He, Ar and H2 at diferent pressures ($p'=P/P_{critical},$)\cite{knuth_average_1999}. the curves represents the regions where the supersonic expansion cam be done and the temperatures (in Fig dimensionles)that each gas should have in order to achive clustering and cooling.\cite{knuth_average_1999}

\begin{figure}[hbtp]
 \centering
  \includegraphics[width= 8 cm]{../Images/dimensiones isentropic diagram.png}
 \label{img:t-s_He}
 \caption{Dimensionless phase diagram for He, H2 and Ar. where T is
dimensionless $T'=(T -T_{tp})/(T_{cr}-T_{tp})$, same as entropy $S'=(S-S_{cr})/\Delta Stp$ and $x=fraction$ of
the fluid in the gaseous phase, where the subscripts cr and tp refer to the
critical point and the triple point, and $\Delta S$ is the entropy change for vaporization. The curves are
drawn as guides to the eye not exact measuments, taken from \cite{knuth_average_1999}.}
 \end{figure}
  

There is no mathematical approach of the physics behind this cooling expansion but usually, Raleigh scattering measurements in combination with an empirical scaling law \cite{hagena_cluster_1972} are used to estimate the mean cluster size giving a certain degree of control over the cluster size distribution by adjusting the nozzle width and the source pressure. The droplet size distribution during supersonic expansion in the follows a log-normal distribution of the form \cite{harms_density_1998}.

\begin{equation}
p(N) = \frac{1}{\sqrt{2\pi}N \sigma} \exp  \left[- \frac{(ln(N/N_{0})^2}{2\sigma^2} \right]
\end{equation}
Where \textit{N} is the number of atom in the cluster, $\sigma$ is the distribution width and \textit{$N_{0}$} is the most likely numbers of atoms. Following it give a mean value.
\begin{align}
\bar N = \exp  \left(\mu+\frac{\sigma^2}{2} \right)
\end{align}
With a half width maxima of \cite{harms_density_1998}
\begin{align}
\sigma N_{\frac{1}{2}} = \exp \left( \mu - \sigma ^2 + \sigma \sqrt{2 ln(2)} \right) - \exp \left(  \mu - \sigma ^2 - \sigma \sqrt{2 ln(2)}  \right)
\end{align}

\begin{figure}[hbtp] \label{fig:ExpRegim}
\centering
\includegraphics[width= 8 cm]{../Images/expansion_regimes.PNG}
\caption[Phase diagram for Expantion regimens]{Expansion regimes. Pressure-Temperatur phase diagram for $^{4}He$ for Nozzle beam expansions starting from a stagnation of 20 bar and a temperatures. As dicusse, quialitatively different behaivors are shown for the regime I - II and II where  starting in the gas phase,  near the phase trnasition respectivelly. taken from \cite{buchenau_mass_1990}. }
\end{figure}

As show in Figure \ref{fig:ExpRegim} The conditions in the He (pressure, temperature and nozzle size) in the free expansion will determine the characteristics of our final He beam. From Here three main regimes can be define.

Regime I or sub-critical expansion, begins in the gas phase and leads to droplet formation via condensation. this is the case of most expansions since the pressure are located below the critical pressure $P_{c}$.
Regime II, also called as critical expansion, is basically  and interminable regime that includes all trajectories which are near the critical point, leading to random expansion and difficult control of the beam due the large fluctuations in density.
Regime III, the  supercritical expansion, starts at low temperatures where the He stops behaving as an ideal gas, expecting flashing or cavitation  breaking up the liquid drops jet. \cite{buchenau_mass_1990}

super-critical and sub-critical regimes have been studies  in the last several years and  are clearly identified in the resulting size distributions. Figure \ref{img:dropletSize} shows that supercritical expansion forms large droplets (usually between $20-100 nm$ diameter) while a sub-critical expansion is suited to generate small droplets (around $5-10 nm$).  A simple relation that can be done to calculate the size or number of atom in a Custer is using. 

\begin{equation}
r=N_{1/3} * \rho A
\end{equation}

Where $r$ is the radius of the beam, $\rho$ is the density, in thhis case of He $\rho =0.0022 A $ \cite{stringari_systematics_1987}, but this approximation is not exact due the variation Of He density at this temperatures. As expected in both regimens for creating larger helium nano droplets, higher helium pressure and lower nozzle temperature are used. For our experiment a $5 \mu m$ nozzle was used at temperature oscillating between $11-15 K$, at pressure of $30 to 50$ bar.

\begin{figure}[hbtp]
\centering
\label{img:dropletSize}
\includegraphics[scale=0.4]{../Images/sizes_regimen.PNG}
\caption[Expantion droplets Regimens]{Sizes of the $^{4}He$ droplets  as a function of nozzle temperature T and  pressures, based on \cite{toennies_spectroscopy_1998}, using a $5 \mu m$ nozzle. The sub and super critical regimes are clearly diferenciated. Taken from \cite{stienkemeier_spsectroscopy_2006}}
\end{figure}


\subsubsection{Composite Clusters}

We can define a composite cluster or doped cluster, as a atom  bulk of one material that contains one or more different atomic elements. The main interesting properties in non doped clusters are usually set as a function of its size, but for doped clusters, the interaction between the elements creates new degrees of freedom that makes more complex its behaviour. For example, the new composite will have different structural properties due the spatial distribution of the species. Hence, composite clusters exhibit a more diverse behaviour and offer more opportunities to study different characteristics of the material.

The first problem to overcome in composite cluster  is how to create them. Two techniques can be used. The first one, is the co expansion of a previously mixed gas \cite{tchaplyguine_variable_2004} or the He cluster is produced first and then crossed with an atomic beam of the doping species.

The first technique  involves several technical problems, depends on possible interactions between the elements, the condensation ranges of the bulks and even in the  affinity  of the materials. One of the most used techniques, and the one used in this study is the one called pick-up technique \cite{gough_infrared_1985}. The idea is simple, as well as a snowball on its way downhill collects or pick-up more snow. The He cluster, after being directional selected through a Skimmer,  passes through a doping cell with a dopant gas at low densities ($10-2 Pa$)  \cite{stienkemeier_spectroscopy_2006}. As a result, the gas atoms that are along the droplet cross sections will be captured by the beam and travel with it. The probability for helium droplets to collect $k$ atoms or molecules via inelastic collisions depends on the length of the oven cell $l$, the cross section of the droplets $\sigma $, and the particle density inside the cell $n$. As $l$ and $\sigma $ remains constant, varying the density in the doping cell can  regulated the abundance of $k$, following the Poissonian statistics.

\begin{equation}
P_{k}(l,n,\sigma)=\dfrac{(ln\sigma)^{k}}{k!} e^{(-ln\sigma)}
\end{equation}

Two important properties of these relation can be infer. First the maxima of different cluster sizes  are equidistant, $n_{max}=\dfrac{k}{l\sigma}$ and second, the exponential function in equation  becomes  nearly  one for  small  particle  densities \cite{bunermann_modeling_2011}.

Every pick-up process leads to an energy transfer to the droplets. As the dopant rapidly cool down to their, it means a transfer of energy to the He causing an  evaporaton of  helium atoms to keep the temperature unchanged. This He evaporation or "shrinkage", leads to a decrease of the cross section of the droplet and the probability to collect a further particle is  reduced. The involved energy is composed of the following contributions \cite{bunermann_modeling_2011}.

\begin{equation}
E=\langle E_{kin}\rangle + E_{in} + E_{binding} + E_{cluster}
\end{equation}

where

\begin{equation}
\langle E_{kin}\rangle \approx \dfrac{3}{2}k_{b}T + \dfrac{1}{2} m v^{2}
\end{equation}

is the kinetic energy of the droplet depending on it mass and velocity and temperature in the gas cell.


\begin{figure}[hbtp]

\centering
\includegraphics[width=14cm]{../Images/He_evaporation (2).png}
\caption{animation of the He creation , doping and evaporation. From left to rigth, we see the He droplet production; after been released by the supersonic jet, the Cluster formation, the pickup process of the dopant and finally the He srinking process.}
\label{fig:shrink}
\end{figure}

At a certain energy entry, the complete droplet evaporates if to may dopping acces to it. With $E_{kin}$ the average kinetic energy, $E_{in}$ the internal. Several studies have studied the $E_{binding}$ with $^{4}He$, given a broth number of materials to work with.. it also importat to take into acount that the binding energy include the cluste dopant binding as well as the dopant$-$dopant relation.\cite{toennies_spectroscopy_1998}. Acommung energy bounding for example $Xe-He$ is arround $26.9 meV$\cite{lewerenz_successive_1995}, or $He-H_{2}O$ is about $0.1 eV$ \cite{lewis_helium_2014}. A more detailed table of all the energy bounding energy used in this study can be found in the appendix.



\subsection{Neon Droplets}

$Ne$ is the second lightest inert gas with atomic number 10, it have 3 isotopes, the $^{20}Ne$ with more than $90\%$ of abundac, folloowed by $^{21}Ne$ and  $^{22}Ne$, all of them  can be found stable in nature \cite{meija_atomic_2016}. As a nobel gas, it shares most of the properties already mentioned for He, excepted  for its superfluidity.  It have a a quite large Ionization potential for the firs electron at $Ip=21.56 eV$, what makes it quite suitable to use with strong fields lasers, as a matrix, because at low intensities it will not interact (get highly ionized) with the laser field, so Dopants can be carried in a non interactive way.

At extreme temperature, Ne is solid as shown in the graphic \ref{img:Nephases}, althougth it triple point and solidification limit allow it to be used at extream low temperatures too. Its triple point is arround $T_{p}=24K$ \cite{young_phase_nodate} and have a rather high initial Ionization Potential. 

\begin{figure}[hbtp]\label{img:Nephases}
\centering
\includegraphics[width= 8 cm]{../Images/Ne_temp_phases.png}
\caption{on the left. Neon phase diagram. taken from \cite{young_phase_nodate},on the right, $T-S$ phase diagram of Ne. The critical point is located at $T_{c}= 44.49 K$ and a molar entropy of $S_{c}=30.76 J/(mol K)$. The dashed lines represent regions. Taken from \cite{christen_supersonic_2010-1} }
\end{figure}

Ne cluster also provide an ideal medium for quemichal reactions as solvation effect and heterogenius chemestry at a miscroscopic level \cite{gough_infrared_1985}. With a proper regulated pick-up system teh reactants are deposited in a controlled way in the cluste and the cluster becomes a nanoreactor. in the same way, the low temperatures at the Ne the reactants will dicresse the ere degrees of freadom and act in a more basic and predictable reactions. \cite{gaveau_reaction_2001}.

 The conditions for creating Ne clusters are quite similar to the ones explanied above, also well explained in the famus Hagena law \cite{hagena_cluster_1972-1}, for different temperatures and nozzels. Several studies have been realized on the characterization of $Ne$ clusters have been achieved. One example is  by \textit{R. von Pietrowski et all}, on contrary to the He clusters, on $Ne$ is important to work at temperatures and pressures far from its solidification point. At extream low temperatures, diferences in preasure leads to a change on the size clusters, the higher the preassure in the nozzle the bigger the droplets. As example in \textit{Pietrowski} work,, was shown that small droplets, $N=300$, been $N$ the number of atom in the cluster, are in a "liquid" stade but for bigger droplets solidification is present. In addition, the location of the doppant also will be affected drastricly by these sizes changes. When the droplet is on a "liquid stade" the dopants atoms are more free to move to the center and arrond the cluster, but in a more dense droplet, this dopant will tendt to stay in the surface of the same. \cite{von_pietrowski_fluorescence_1997}.

\begin{figure}[hbtp]
\centering
\label{img:t-s_Ne}
\includegraphics[width= 8 cm]{../Images/T-s ne phase diagran.png} 
\caption{on the left. Neon phase diagram. taken from \cite{young_phase_nodate},on the right, T$-$S phase diagram of Ne. The critical point is located at $T_{c}= 44.49 K$ and a molar entropy of $S_{c}=30.76 J/(mol
K)$. The dashed lines represent regions. Taken from \cite{christen_supersonic_2010-1} }
\end{figure}

Same as the T-S representation of He,Fig \ref{img:t-s_He},  fig \ref{img:t-s_Ne} shows the  isentropic processes as simple vertical trajectories. One advantage in this color plot (color available on-line) is the visibility of the two-phase region where condensation may take place. The dashed lines represent isobar lines at $p= 100,1000,1000,10000,$ and $100000$ $Pa$ from left to right respectively. SOn one hand, upersonic expansions which originate in the vapor phase at a very low source pressure, equivalent to a comparatively large stagnation entropy $s_{0}$, will not reach this region. On the other hand, for negative entropies the solid state is always reached and for  low temperatures ad relative small entropy the liquid state is the predominant. 
In contrast, supersonic jet expansions which originate at high source pressures will arrive at the saturation curve. Thermodynamically, condensation is feasible at the gas-liquid phase boundary. Accordingly, isentropic expansions reaching the binodal line (Magenta point in the peak of the Vapor-liquid phase) might be expected to yield both uncondensed particles and condensed species such as clusters and droplets.
Hence, for sufficiently high pressures it is shown from this phase diagram to expect always a gaseous beam; this is irrespective of the initial reservoir temperature.


\section{Cluster-Intense Fields  Interaction}

To understanding of the interaction atoms-fields have been study broadly in physics since Einstein Photoionizasion Theory \cite{einstein_uber_1905}, that gives a base on all the quantum electrodynamics theory. The basics unders this theory is the behaivour of ligth as a electromagnetic field where the electron as a bounded charge in the atom can be afected.  This quantum dynamic theory is well understood since 1957 for small atoms, with one, two or few electrons \cite{a._bethe_quantum_1957}, but still big molecules and atoms have been challenging scientific for years. In this chapter we will give a brief introduction to the photoionization process, explaining at the same time multi-photoionization and tunnelling precesses, so we can finish with a more detailed presentation of Strong field interaction with clusters and the Keldish theory.


\section{PhotoIonization for single atoms}

The process of photoionization describes the leaving of an electron from its bound state  into the continuum by interaction with electromagnetic field radiation\cite{berkowitz_photoabsorption_1979}. The atomic bounded electrons while going through an electromagnetic field, in our case the laser beam,  can absorb enough  energy to get excited and fly away from the nucleus. A bound electron only can escape from an atom by absorbing photons its energy exceeds the binding energy of an electron \cite{einstein_uber_1905}. When the photon energy of the laser is smaller than the ionization potential of the target, the electron can absorb two or more photos in the ionization process, this is called Multi photon ionization (MPI). Another possible process is called, tunnelling ionization, where due the quantum mechanical properties of the electrons under certain conditions absorbs enough energy enough to be in an above threshold regime, due it quantum dynamic properties it can escape from its bonds via tunnelling.

There is a variety of theoretical approaches to describe the interaction of  laser fields with atoms. The Hamiltonian of the system of $N$ particles (ions and electrons) with pairwise Coulomb interactions under the action of an external time-dependent electric field has the form:
\begin{equation}  \label{eq:hamiltonian}
\centering
H = \displaystyle\sum_{1 \leqslant i \leqslant N}^{} \dfrac{P_{i}}{2m_{i}} + \displaystyle\sum_{1 \leqslant i < j \leqslant N}^{} \dfrac{q_{i}q_{j}}{\mid r_{i}-r_{j} \mid} + \displaystyle\sum_{1 \leqslant i \leqslant N}^{n} q_{i}r_{i}\varepsilon(t)
\end{equation}

where $ r_{i,  p_{i}} $ and $ q_{i} $ are the coordinates, momenta and charge of the particles, including the interaction between the classical electric field and $ \varepsilon(t) $ where \cite{mikaberidze_atomic_1981}

\begin{equation}
\varepsilon(t) = \varepsilon_{0} e_{z}cos(\omega t + \varphi)
\end{equation}

The process that drives ionization can be divided on two regimes, a quantum electrical regime and a classical one. \cite{karnakov_strong_2009}. Equation \ref{eq:hamiltonian}, use the non-relativistic approximation and neglect contributions from magnetic fields. The classical description of the laser field is a good approximation for intense enough pulses, otherwise, quantum electrodynamics description is necessary.

An electron in the initial level with energy $E_{i}$ can absorb an photon with energy $\hbar \omega$ leading to final transition where $E_{f}-E_{i}=\hbar \omega$, when the energy of the photon is larger than the bounding energy, or the Ionization barrier the electron is free with a the remaining kinetic energy $E_{kin} = \hbar \omega - I_{pot}$ \cite{becker_vuv_1996}.
In classical mechanics the probability of the energy transition depends directly on the cross section $(\sigma)$ of the electron and the field. However,  in  quantum  mechanics,  the photoionization cross section is related to the its transition probability between the initial and the final state given by Fermi’s golden rule

 \begin{equation} 
 \label{eq:transitionprobability}
W_{|i\rangle \rightarrow |f\rangle} = \frac{2\pi}{\hbar\hbar} |\langle f|H|i\rangle|^{2} \delta(E_{i} - E_{f}-\hbar\omega)
 \end{equation}
 
 \begin{equation}
 \label{eq:crosssecQ}
 \sigma(\hbar \omega) = \frac{2\pi}{3} \alpha a_{0}^{2} \hbar \omega |\langle f|r_{n}|i\rangle|^{2}
 \end{equation}
 
When eq. \ref{eq:transitionprobability} is the transition probability of one electron to jump from initial state $i$ to final state $f$, where $H$ is the Hamiltonian operator. Eq. \ref{eq:crosssecQ} is the consequent cross section considering only the dipole part of the interaction Hamiltonian, where $\alpha$ is the fine structure coefficient, $r_{n}$ is the position operator of the electron $n$ \cite{fermi_quantum_1932}.

The energy photon needed to ionize an atom, is directly proportional to the energetic distance between the electronic states and the ionization threshold. For  states closer to the ionization potential a  VUV photon can be enough to free an electron but for inner electrons  higher photon energies are required, varying from several tens eV to the order of several tens of keV, needing radiation sources at shorter wavelengths such as XUV to X-rays.\cite{becker_vuv_1996}

After photoionization is done, the electronic structure of the atom needs to rearrange via, due to the vacancy left by the ejected electron. Two relaxation processes can happen during this time. An electron from the outer shell will decay and replace the freed one, therefore the energy difference of the needs to be released in the form of a fluorescence photon or Auger electron. On one hand, in case of a fluorescence decay the ionic state of the target does not change, since no additional electron is released. On the other hand, the Auger decay is a non radiative relaxation process, where a second electron is released from the Coulomb potential of the ion.

In example. As shown in fig \ref{fig:augerfluorec}, if an photon  with energy $\hbar\omega > E_{bin}$  ionized an electron, this will leave the atom lifting a gap. An electron in the higher levels will replace the outer one, leaving an excess of energy. The outcome will be a fluorescence process with $E_{flu} = E_{in}- E_{out}$ or , the Auger $e-$, if $ E_{in}-E_{out} > E_{bond}$ and this electron can also escape the atomic Coulomb potential \cite{schmidt_electron_1997}.

\begin{figure}[hbtp] 
\label{fig:augerfluorec}
\centering
\includegraphics[width=6 cm]{../Images/text6418.png}
\caption[Relaxation processes for photoionization]{two example on the relaxation processes. On the left, A photon ionized an electron and the Electron $E_{in}$ replaced, expelling a fluorescent photon in the process. on the right, the energy released by the replacement electron is enough to make another electron in the outer shell to also go to the continium, Auger electron. Taken form \cite{rafipoor_two-color_2017}}
\end{figure}

\subsection{Multhiphoton and tunnelling Ionization}

Ionization is also possible even when one photon energy is lower than the binding potential. Laser fields with intensities below $I \leqslant 10^{14}W/cm^{2}$ are not strong enough to change the binding potential of an atom significantly \cite{rhodes_multiphoton_1985} and is when multiphoton Ionization takes place (MPI).  MPI is the simultaneous absorption of several photons to overcome the ionization barrier. The way MPI occurs depends on the laser frequency and intensity. When the intensity is much lower than the characteristic atomic resonance, MPI occurs via transitions through virtual states. Ionization by several photons at low laser intensities can be realized by the so-called resonance enhanced multiphoton ionization (REMPI)\cite{mainfray_multiphoton_nodate}.  Ionization by a REMPI process takes place in two steps
First, a resonant excitation by one or more photons takes place on an electron state of the atom. In the second step, this electron state is transformed into a virtual state, to an upper state until the electron is excited by spontaneous decay. So for example the total energy absorbed by an election until it gets ionizes is $n * \hbar\omega > I_{pot}$ where $n$ is the number of photons absorbed until it actually have enough energy to overcome the potential $I_{pot}$

For Laser intensities $I > 10^{14}W/cm^{2}$, higher intensities and lower frequencies, tunneling ionization (TI) is more likely to occur.  In this case, the binding potential of the atom get is strongly affected by the electric field of the laser. Around the peak of the electric field the  potential gets narrower, and the electron in the outer states get closer to the bidding barrier, allowing the electron to tunneling through the confining potential to the continium  \cite{griffiths_introduction_2013}. TI is inherently a quantum process. The bending of the Coulomb potential becomes by the superposition of the coulomb potential and the laser field. Therefore TI must occur when the time of the ionization is shorter than a laser oscillation cycle\cite{berkowitz_photoabsorption_1979}. Based in the same principle, when the Laser field becomes so strong to lower the binding potential that separates the highest electron level, then the electrons in this state become free electrons. This process is called barrier suppression ionization or BSI\cite{krishnan_doped_2011}.

In the fig \ref{img:ionizationprocess}, we present a sketch of the 3 possible ionization processes explained above. On the left we present a simple ionization process where a photon with energy $E_{phot} = \hbar\omega$ is higher than the potential barrier. In the center a MPI process is shown, $n$ photons excite the inner-shell electron, exiting it through virtual level until it finally has enough energy to be free to the continium. Finally, on the left a TI happens. Here the coulomb potential barrier is affected by the laser files bending, the outer shell electron  gets closets to it until it tunnels \cite{rafipoor_two-color_2017}.

\begin{figure}[hbtp]
\label{img:ionizationprocess}
\centering
\includegraphics[width = 8 cm]{../Images/photoionization2.png}
\caption[Ionization regimes]{ On the left is the sketch of a single photon ionization process, where a photon with energy $E_{phot} = \hbar\omega$ is higher than the potential barrier $I_{p}$. On the center the MPI process,  inner-shell electron absorbs $n$ photons, getting excited through the electronic levels (reals or virtual)  until it reaches the continuum. On the left the BSI Process, the coulomb potential barrier bends by the laser fields, been lower than the outer shell electron state, the electrons can scape easilly. based on \cite{rafipoor_two-color_2017}.}
\end{figure}


As explained the Intensity of the external field plays an important role in the ionization process. A rather easy way to differentiate when each process needs to be taken into account is provided by the Keldysh parameter\cite{keldysh_ionization_1965}.

\begin{equation}
\gamma_{k}=\sqrt{\dfrac{I_{p}}{2U_{p}}}
\end{equation}

Where $\gamma_{k}$ is the Keldish parameter, $I_{p}$ is the atoms ionization potential and $U_{p}$ is the ponderomotive potential defined as:

\begin{equation}
U_{p} = \dfrac{e^{2}E_{0}^{2}}{4m_{e}\omega_{0}^{2}} \propto I \lambda^{2}
\end{equation}

where $m_{e}$ is the mass of the electron, $\omega_{0}, \lambda, I$ and $E_{0}$ are the frequency, wavelength, intensity and the peak of the electric field of the laser pulse. On one hand, when the Keldish parameter is higher, $\gamma_{k} \gg 1$ MPI regime is considered. On the other hand, the $\gamma_{k} \ll 1$ describes the TI interaction.

\subsection{kelysh Theory}

In this section we will give a brief introduction to keldysh theory and the main repercussions for our work as the ionization rates. We were based this subchapter in the work of Keldy et all, \cite{keldysh_ionization_1965}, and the papers review of the theory by \cite{popruzhenko_keldysh_2014} and \cite{karnakov_strong_2009}. For a deeply explanation we recommend the reader to reference this works.

The keldysh Theory, also known as the Keldysh–Faisal–Reiss theory (KFR),  is well used for the description of quantum process induced by intense laser radiation. The applications and advantages of Keldysh formulation in many-body theory among  can overcome from, treatment of systems away from thermal equilibrium (with or without presence of external fields), solutions in  super symmetry methods of systems with quenched disorder to the  calculation of the full counting statistics of a quantum observable\cite{kamenev_introduction_nodate}.

According to the Keldysh Ansatz, the transition probability amplitude between an atomic bound state and the continuum by the value of the photoelectron momentum $p$ measured at the detector is given by: \cite{popruzhenko_keldysh_2014}.
 \begin{equation}
 M_{k}(p) = -\dfrac{i}{\hbar} \int_{\inf}^{+\inf} \langle \Phi_{p}\mid  V_{int}(t)\mid \Phi_{0} \rangle dt
 \end{equation}
Where $M_{k}$ denotes the Keldys transition probability, $\Phi_{0}$ is the bond state wave function unperturbed and $\Phi_{p}$ is the canonical momentum, equal to $p$, also known as the Volkov Funtion, and $V_{int}$ is the electron field interaction operator. 
If the amplitude of ionization $M_{k}(p)$ is known, the differential probability to find the photoelectron in the elementary volume near the momentum $p$ is given by the momentum distribution of the photoelectrons 
\begin{equation}
dW(p)=\mid M(p)\mid^{2} d^{3}p
\end{equation}
 Giving a total probability of
 \begin{equation}
 W= \int \mid M(p)\mid^{2} d^{e}p
 \end{equation}
Meaning that, for enough long pulses, containing a large number of optical periods so that its electromagnetic field is close to a periodical function of time close to the initial, it is physically more appropriate to use probabilities per time unit (rates) instead of time-integrated values.

\subsection{Ponderomative energy}


As soon as an electron is released into the continuum, it is  under the influence of the external laser field. A description of the energy that it acquires during this interaction is given by the ponderomotive energy (PE).

\begin{equation}
U_{p} = \dfrac{e_{2}E_{a}^{2}}{4m \omega^{2}}
\end{equation}

Where $m$ and $e$ is the electron mass and charge, $E_{a}$ and $\omega_{0}$  amplitude and frequency of the electric field respectively. The formula of the ponderomotive force can be easily  derived as shown in \cite{protopapas_atomic_1997}\cite{connerade_highly_1998}. Let`s consider a polarized electric field (in a.u).

\begin{equation}
E=\widehat{z}E_{a}sin(\omega_{0} t) 
\end{equation}

Considering only the $\widehat{z}$-components so we can avoid the vector sign. By classical mechanics we have.

\begin{equation}
p(t)=-\int_{t_{0}}^{t} E(t\prime)dt\prime = \dfrac{E_{0}}{\omega} (cos(\omega_{0} t)- cos(\omega_{0} t_{0}))
\end{equation}

The term on the left of the parenthesis is known as the time varying Quiver terms, and on the one on the right, reefers to the drift motion. 
Expressing the fields in terms of vector potential we will have

\begin{equation}
E(t) = \dfrac{\delta A(t)}{\delta t}
\end{equation}

\begin{equation}
p(\infty) = A(t_{0}) = -\int_{-\infty}^{t} E(t) dt = \dfrac{E_{0}}{w} cos(w_{0}t_{0})
\label{eq:pmax}
\end{equation}

in the case where the pulse  duration is big $t \longrightarrow \infty$ the $p + A(t) = 0$. This means that the momentum acquired by the electron will depend on the phase it is realized $wt$. Since the electron can be unbound in any phase of the laser pulse, will have an average kinetic energy described by

\begin{equation} 
U_{p} = \dfrac{1}{2\pi} \int (-\dfrac{E}{w} cos (wt))^{2}d(wt) = \dfrac{E^{2}_{0}}{4w^{2}_{0}} = \dfrac{p_{max}}{2}^{2}
\label{eq:pondeenergy}
\end{equation}

The pondemorotive energy also gives the maximum momentum that an electron can acquire (eq. \ref{eq:pmax}), given at the maxima. The $\omega t$ phase relation, defines what it called\textit{the three step model} showed in figure \ref{fig:ponder} . The first step corresponds to $\omega t < \pi /2$ where the laser field is suppressed, and as explained above, TI or BSI can take place. The second step, is where $\omega t > 3\pi /2$, on contrary step 1 the potential barrier is enhanced, electrons in the continuum that was winning kinetic energy are caught by the potential again, being driven  back to the atom. Finally the step 3 at  phase  $\omega t = n* \pi$, for $n 0 1,2,3...$. At $n=2$ it is called "recollision process" of the electron. Where the electron can be caught by the potential again, and the excess of energy release another bound electrons, depending on the kinetic energy necessary \cite{krishnan_ignition_2012}

\begin{figure}[hbtp]
\centering
\includegraphics[width=12 cm]{../Images/ponderomotive steps.png}
\caption[Ponderomotive 3 steps]{Recollision  process at the three step model. Taken from \cite{krishnan_doped_2011}}
\label{fig:ponder}
\end{figure}


If we transform the eq. \ref{eq:pondeenergy} to laser intensities we will have $U_{p} = 9,33*10^{-14}I[W*cm^{-2}] * \lambda^{2}[\mu m]$. For a MIR-pulse with intensities $\sim 10^{14} [W/cm^{2}]$ and $\lambda \sim 3200 [nm]$ we have electron energies between one and a hundred of eV.

\subsection{Cluster Ionization}

Until now, we have shown one of the most known ionization processes for single atom. But on cluster ionization, the dynamics are quite different and need a bit more detail. Clusters are combinations of atoms or molecules which, depending on their species, are held together by Van der Waals forces, ionic bonds or metallic bonds. In this explanation we refer only now and on to He cluster, which mainly are just affected by Van der Walls attraction. The investigation of clusters is of scientific interest in many studies, a cluster can be generated in a wide range of sizes ($10^{2}$ to $10^{7}$ )\cite{stienkemeier_spectroscopy_2006}, having a radius significantly smaller than the MIR laser pulse wavelength used on them, meaning that all its atoms are equally affected by the pulse filed, in other words it penetrates all over the cluster.

The first problem that He cluster faces is auto-ionization. Being a rare gas, its ionization potential is higher than many of its doping molecules used. For example, under MIR lasers He clusters need $I > 10^{15} W/cm^{2}$ , so TI or BSi is not the main process at beginning of the plasma generation. Other interactions have to be explained in order to describe the process properly. This section we will be based on \textit{Saalaman et. all} work \cite{saalmann_mechanisms_2006} and \textit{Grüner et. all} \cite{gruner_femtosekundenspektroskopie_2013}, for this purpose, we will divide the process in three phases or stages.


The first stage, called \textit{“atomic ionization”}, the doping atoms are ionized independently of each other by the electric field at the leading edge of the laser pulse, it occurs mainly through inner ionization, especially on TI or BSI. The resulting free electrons acquire positive kinetic energy and have two options, leaving  the cluster or they stay inside the cluster attracted to its positive ion core. After the first stage, the cluster nanoplasma is \textit{ignited}, consisting of ions and quasi-free electrons, electrons that are free to travel inside the cluster volume but still not into the continuum \cite{last_quasiresonance_1999}. 


The second stage is the  \textit{nanoplasma expansion}. During this stage the cluster is still interacting with the laser field, acquiring, by a large number of processes, energy in atoms and electrons. Ions are further created by a combined force of the laser and other ions, the \textit{ionization ignition}\cite{gruner_femtosekundenspektroskopie_2013}. Quasi-free electrons oscillate, driven by the laser pulse and are heated to high temperatures. The heating becomes extremely efficient when the collective oscillations of quasi-free electrons become resonant with the laser pulse, Triggering a cascade reaction to more outer ionizations,  freeing the remaining electrons in the cluster, this process is called \textit{plasma resonance}\cite{saalmann_mechanisms_2006}.


After the laser pulse is over, the last stage starts. The ions continue to expand, in consequence, the radius rises as same as its cluster potential becomes smoother. So, it is easier for the highly energetic quasi-free electrons to leave the cluster, forming a coulomb explosion that destroys the cluster in a beautiful ions-electrons cascade. This process was first describe by \textit{Ditmire et. all} \cite{ditmire_interaction_1996} combining high energetic collisions with cluster resonance absorption.

The graphic \ref{img:clusterpotential} shows on the left how the cluster potential is composed by the Van der Walls potential and atomic forces of the different atoms that compose the cluster. On the central images, additional to the atomic biddings, the electrical force due the ions in the cluster increase the potential, the laser pulse is still on and the quasi-free electron will gain energy while they are in this. Finally, on the right the laser field is off, the electrons have fled away and the cluster ion have been repelling each other so the potential is reduced to the minimum(just atomic interactions.)

\begin{figure}[hbtp] \label{img:clusterpotential}

\centering
\includegraphics[scale=0.35]{../Images/clusterpotential.PNG}
\caption[Cluster potential regimes]{Cluster potential regimes. On the left, the atomic ionization starts the plasma formation. On the center, The quasi-free electrons auto-ionize the cluster, increasing the potential barrier and  gaining energy due the laser field so a coulomb explosion can take place. On the right, The Coulomb explosion is finished, the potential is driven to it minimum and all the electrons and ions are ejected. Taken from \cite{wabnitz_multiple_2002}}.
\end{figure}

\subsubsection{Cluster expansion}

Depending on the droplet size and the laser intensity, the Cluster can expand in two different ways. If the laser intensity is rather high and the droplet is small, a Coulomb explosion can occur. On the contrary, if the laser is not intense enough or the droplet is too big, a nanoplasma can be generated, therefore an hydrodynamic expansion will take place. 
Two forces are really important during the cluster expansion. Both act on the cluster during the phase two and three(during and after the laser pulse). The first, is the force associated with the free electrons with high kinetic energy. These hot electrons expand and pull the low energetic electrons and heavy ions on its pad  \cite{ditmire_interaction_1996}. The other force acting on the cluster is due to the inner cluster charge itself. The hottest electrons in the cluster will have a mean free path large enough so they can free stream directly out of the cluster, and, if the electron’s energy is large enough to overcome the space-charge buildup on the cluster, they will leave the cluster altogether. If the charge buildup is sufficiently large, the cluster will undergo a Coulomb explosion \cite{haught_formation_1970}. According to Madison et all, a time scale for the laser pulse duration  where the coulomb explosion can take place should be  closer or lower to the femtosecond regime, depending  on the element composing the cluster. \cite{madison_role_2004}. Based on the laser power available on modern laser pulses, the same studies present that electron after a coulomb explosion can get kinetic energy up to 6KeV.


When the intensity is not enough to make the atomic bonds to break, the  electrons remain in the cluster forming a  hydrodynamic expansion as a result of a conversion of electron-thermal energy to direct kinetic energy \cite{erk_nobel_2009}. The effects that the expansion has on the electron temperature can be calculated by equating the rate of change of radial kinetic energy from the thermal contribution with the rate of change of thermal energy within the cluster. When this condition is fulfilled. The electron can present a resonance condition in the cluster, traveling in the space-charged forces formed by the plasma, winning enough kinetic energy until all the system collapse.

\begin{figure}[hbtp]

\centering
\includegraphics[width=14cm]{../Images/cluster_regimes_2.png}
\caption{sketch of the Coulomb explosion for a doped cluster due the excitation of a pulsed laserfield. At the beggining of the process the laser ionized the droplet, until some femtoseconds (up to 500fs) after, the system colapce and result into a coluomb explotion.}
\label{fig:columbexplosion}
\end{figure}


Although the two models are different in each regime, for example, at low kinetic energy or the beginning of the pulse, the Coulomb explosion produces less ions for low energies compared to the product on hydrodynamic explosions. Further, the number of  high energy ion the coulomb explosion can create (although in less quantity) tent to be hotter ions too.
We have to take into account that even the two processes are described for different laser regimes. Both processes can happened in parallel, but at certain energies is clear, that one or the other will be the responsible at the end, for the collapse of the system.

\section{VMI spectrometer}

A Velocity-Map-Imaging or VMI spectrometer is used to investigate velocity distributions of charged particles, which are generated in a defined volume, typically by photoionization. The distribution of the particles, on the other hand, should be the object of the measurement and therefore also have an influence on the results. 
A laser generates charged particles by photoionization in the interaction region, which are accelerated by means electric lens created by two electrodes. Typically, this is a system consisting of an Microchanel plate or MCP  and a phosphor screen, which is observed by means of a CCD camera on top. Electron or Ions are created from the atomic photoionization, these particles are accelerated by an applied voltage and generate further electrons. Once these arrive to the MCP an electronic avalanche occurs, been able to detect even single electrons. The electrons hit the phosphoscreen igniting photons that a focussed CCD camera can detect.
The VMI used in this experiment satisfies this requirement is the Eppink-Parker-Design \cite{eppink_velocity_1997}, the lower  electrode is called Repeller followed by the extractor, ground electrodes and a E-lens before hitting the MCP array.  All electrodes have a circular layout and are about $1 mm$ thick.  The extractor and ground electrodes have a concentric hole where the particles pass through.
 

\subsection{Velocity distribution in VMI}

Based on the assumption done above where the VMI, we assume that the charged particles come from a point in space emitted between the Repeller and the extractor. Furthermore, we presume that the energy generated by the excess kinetic energy of the particles obtained during the ionization process, small in comparison to the is the energy they receive from the electric field.
The initial conditions of the charged particles are determined by the ionization volume and the initial velocity vectors $v_{i} = (v_{x}; v_{y}; v_{z})$, whose distribution is to be measured. The particles are measured by the E-field in the direction of the spectrometer axis of the planar detector surface
Where they accelerate, depending on their initial velocity vectors. If one starts from an isotropic initial distribution of the velocities, particles with differing initial velocity vectors can finish in the same point on the detector, as shown in fig \ref{ing:vmiVelcDist}. This results in a loss of information due we are trying to get some data of a 3D space from a 2D projection.

\begin{figure}[hbtp]
\label{ing:vmiVelcDist}
\caption{Representation of particle trajectories A, B and C of the ionization volume to the detector plane, which, in spite of initially different velocity the finish on the same point of the detector plane. Withdrawn from \cite{fechner_lutz_aufbau_2011}}
\centering
\includegraphics[width=8 cm]{../Images/cel distrub vmi.png}
\end{figure}
Assuming a cylindrically symmetric distribution $f(r,y)$ along the $y$-axis. If an infinitely far away observant look at the distribution $F$ along the $z$-direction.  He sees the integrated (along the z-axis) sigma
With a basic axis conversion it can be shown that along the Z direction, the projection of the distribution $F$ of the integrated signal respond to.

\begin{equation}
F(x,y)=2\int^{\inf}_{\mid x\mid} \dfrac{f(r,y)r}{\sqrt{r^{2}-x^{2}}}*dr 
\end{equation}

Been F in cylindrical coordinates and $r^{2}=x^{2}+z^{2}$. Also called \textit{Abel-transform}.
But in order to resolve or signal we need the opposite procedure, turn this 2d distribution into the 3d spherical distribution that we assume in the beginning. For this process is called \textit{Inverse Abel-transform}, and it can be shown as.
 
\begin{equation}
F(r,y)=\dfrac{1}{\pi} \int^{\inf}_{\mid x\mid} \dfrac{dF(x,y)}{dx} \dfrac{1*dx}{\sqrt{r^{2}-x^{2}}} 
\end{equation}

The main problem that carries this transformation is that is not resolvable to a non continues distribution, that is the case of the Images recorded by the camera. So a  numerical method needs to be used.
Fortunately, a Software Pbasex, is available to solve this integral. The  idea  is  to  have a basis function $f_{k}$ (in this case Lagrange polynomial ) in the space of distributions $f$.  Their projections $F_{k}$ onto the detector are well known, and the Able transform will be set as:

\begin{equation}
F_{k}(x) = 2\sigma \rho_{k}(x)\lbrace 1 + \sum^{k^{2}}_{1}\lbrace (x/\sigma)^{-2l} \prod^{l}_{m-1}(\dfrac{(k2+ 1-m)(m-1/2)}{m}\rbrace \rbrace
\end{equation}

If every measured image can  be expressed in this basis, then the reconstruction of the F distribution when $k\rightarrow \inf$ can be obtained. A graphical example of this transformation is given in fig \ref{img:Abel} where each color represent a $f_{k}$ projection, where all $f$ are in the same basis, the sum of all the different $f$ will set the final reconstructed $F_{K}$.

\begin{figure}[hbtp]
\label{img:Abel}
 \caption{At the top of the graphic some basic functions $f_{k}$ in different colors, which can be used for the Inverse-Abel transform in the BAsex method. On the bottom, the corresponding  Abel-transformed functions Fk. Taken from \cite{fechner_lutz_aufbau_2011}}
 \centering
 \includegraphics[width=10 cm]{../Images/abel transform pbasex function.png}
 \end{figure}
