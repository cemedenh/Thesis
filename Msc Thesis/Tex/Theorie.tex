% !TeX spellcheck = en_GB
% !TeX spellcheck = en_US 

\\chapter{Theoretical Background}

In this chapter we will present all the theoretical background necessary for the development of this project, from the theori and creation of the He dropletrs to the physics behind the plasma and coulomb explotion procces to the detection thwcniques. In order to guuide the reader in an organiyed way, the chapters are organized ina way that follow the proceses necesari to the performance of the experiment. this means that all the chapters explained in heres occuresa in the ssame order during the experiment.


\section{Helium Nanodroplets}

The combination of cryogenic matrix isolation, discovered in 1954 \cite{whittle_matrix_1954}, and the now well defined properties of Helium ($He$), specially its superfluidity face discovered in 1937 by \textit{Kapitza et. all}\citep{kapitza_viscosity_1938}, have as consequence one of the most powerful and flexible tool in physics, the helium nanodroplets.
Helium nanodrops  have unique properties that makes it  very suitable for the cluster and nanophysics experiments in the last decades. For example, they do not exhibit any optical transitions in the entire infrared, visible and ultraviolet range. They can readily pick up atoms and molecules and  form complexes from the species embedded in their interiors, or on their surfaces and act as a ideal matrix for atom, molecules and clusters isolation. \citep{stienkemeier_spectroscopy_2006}\cite{toennies_superfluid_2004}.
The size of a He cluster can go from of a few thousands up to $10^{8}$ of atoms, and reach temperatures at   ultra cold temperature regime (close to 0.37$K$ \cite{toennies_spectroscopy_1998})\cite{enss_low-temperature_2005}.
Two main advantages of this  cooling properties arise. First,  dopants in the He nanodroplet are set to their absolute vibronic ground states, avoiding all other possible espectra and stablishing the cluster in a specific state, more important, the fast cooling helps to the formation of isomers that are difficult or impossible to generate with other methods \cite{nauta_nonequilibrium_1999}. Second, because the superfluid fase of the He nanodroplets\cite{grebenev_superfluidity_1998}, the bond between dopants and He is weak. Therefore, in contrast to spectroscopy in other matrices with higher temperatures, the optical transitions of many dopants are barely influenced by the He matrix \citep{toennies_superfluid_2004}. 
The theorical part for superfluidity of He will not be part of this section, this imformation is well documented in other sources, and here we was based on ref.\cite{enss_low-temperature_2005} where all this information is well presented to the reader. In the next secction we will dedicate a bigger effor in explain the theorical and thecnical background of the He nanodroplets creation as weel as the physical and thecnical proces to dopped it. 





A helium atom has two electrons. This means that the chemical properties
of this element, everything important is said, because everything else opens up.
directly from this fact:
The electron configuration of the ground state is 1s2. So it has a completed
shell and is therefore a noble gas. Since with helium just the first shell
it is the noble gas with the most strongly bound electrons.
Thus, helium is also the smallest of all atoms. The first excitation energy of helium
is 19.8 eV, the first ionization energy is 24.6 eV, for the ionization of the
second electron one needs additionally 54.4 eV [NIS13]. Helium has therefore in the
complete visible spectrum no transitions from the ground state. Through
the noble gas configuration, helium has a spherically symmetrical electron distribution.
It can hardly be polarized and is the least reactive of all the elements. Neutral
Helium can only form chemical bonds via Van-der-Waals forces, covalent
Bonds do not exist. Therefore, helium forms only very weakly bound
Complexes are helium dimers. Helium dimers are bound with only 0.0001 meV. For complexes
from several helium atoms the binding energy per atom slowly increases
until it reaches the value of bulk helium1 for more than 10 000 atoms, wherein



Understanding the properties of matter starting from the interplay of atoms and molecules
has been achieved to a great deal by the study of small or model systems containing only a
few atoms.  A detailed view on geometric as well as electronic properties has been brought
forward largely through the application of spectroscopic tools.  These tools are continuously
being improved, in particular with the aid of the availability of sophisticated new laser systems,
setting new milestones in terms of repetition rate, power and the time and frequency structure of
ultrashort pulses. On the other hand, the path from small model systems to complex functional
structures is arduous to climb.  In recent years, it has became clear that one approach towards
the understanding of complex structures of atoms and molecules is to start with well-defined
structures  and  well-defined  distributions  of  populated  states.   This  calls  for  spectroscopic
studies at very low temperatures and now a new field emerged dealing with cold molecules or
ultra-cold chemistry. In this regard, spectroscopic experiments involving helium droplet beams
(HElium NanoDroplet Isolation (HENDI)) proved, since their introduction in 1992 [
1
], to be
a versatile method that provides temperatures below 1 K and offers the possibility of studying
well-defined and complex structures of atoms and molecules.  Moreover, the quantum nature
and in particular the superfluid properties of the droplets allow one to investigate these quantum
properties  in  a  size-limited  aggregate  on  the  nanometre  scale.