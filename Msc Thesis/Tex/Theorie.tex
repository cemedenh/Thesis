% !TeX spellcheck = en_GB
% !TeX spellcheck = en_US 

\\chapter{Theoretical Background}

In this chapter we will present all the theoretical background necessary for the development of this project, from the theori and creation of the He dropletrs to the physics behind the plasma and coulomb explotion procces to the detection thwcniques. In order to guuide the reader in an organiyed way, the chapters are organized ina way that follow the proceses necesari to the performance of the experiment. this means that all the chapters explained in heres occuresa in the ssame order during the experiment.


\section{Helium Nanodroplets}

The combination of cryogenic matrix isolation, discovered in 1954 \cite{whittle_matrix_1954}, and the now well defined properties of Helium ($He$), specially its superfluidity face discovered in 1937 by \textit{Kapitza et. all}\citep{kapitza_viscosity_1938}, have as consequence one of the most powerful and flexible tool in physics, the helium nanodroplets.
Helium nanodrops  have unique properties that makes it  very suitable for the cluster and nanophysics experiments in the last decades. For example, they do not exhibit any optical transitions in the entire infrared, visible and ultraviolet range. They can readily pick up atoms and molecules and  form complexes from the species embedded in their interiors, or on their surfaces and act as a ideal matrix for atom, molecules and clusters isolation. \citep{stienkemeier_spectroscopy_2006}\cite{toennies_superfluid_2004}.
The size of a He cluster can go from of a few thousands up to $10^{8}$ of atoms, and reach temperatures at   ultra cold temperature regime (close to 0.37$K$ \cite{toennies_spectroscopy_1998})\cite{enss_low-temperature_2005}.
Two main advantages of this  cooling properties arise. First,  dopants in the He nanodroplet are set to their absolute vibronic ground states, avoiding all other possible espectra and stablishing the cluster in a specific state, more important, the fast cooling helps to the formation of isomers that are difficult or impossible to generate with other methods \cite{nauta_nonequilibrium_1999}. Second, because the superfluid fase of the He nanodroplets\cite{grebenev_superfluidity_1998}, the bond between dopants and He is weak. Therefore, in contrast to spectroscopy in other matrices with higher temperatures, the optical transitions of many dopants are barely influenced by the He matrix \citep{toennies_superfluid_2004}. 
The theory of  He superfluidity will not be part of this section, this imformation is well documented in other sources, and here we are based on ref.\cite{enss_low-temperature_2005} where all theory is well presented to the reader. In the next section we will dedicate a bigger effort on explain the theoretical and technical background of the He nanodroplets creation as well as the physical and technical process to doped it. 

\subsection{He Nano droplets production}

At room temperature, helium is a light inert gas. It is odorless, colorless, tasteless, and after hydrogen, the second most abundant element in the universe.  \cite{enss_low-temperature_2005}. It have a simple 2 atoms structure, exhibing numerous exotic phenomena whose theoretical descriptions are rather complex in many cases, i.e it characteristics of  a quantum fluid. From helium exist  two stable isotopes $^{3}He$ and $^{4}He$.  $^{4}He$ has two electrons, two protons and two neutrons, no nuclear spin and no total spin, pertaining to the bosonic family, while $^{3}He$ with only one neutron has a spin of I = $1/2$ and belongs to the fermions \cite{atkins_liquid_2014}.

The bosonic state $_{4}He$ is specially of interest, at  temperature T$\leqslant$2.8K and under normal pressure has a phase transition from "normal liquid" He-I to super liquid He-II \cite{swenson_liquid-solid_1950}, in which the helium can be described by a Bose-Einstein condensation. Even the fermionic $^{3}He$ exhibits this phase transition at T$\leqslant$0.03K \cite{halperin_properties_1978}.
The superfluidity of He-II, at temperatures close to absolute zero, brings with it some unique features. The essential Properties for this include an almost disappearing viscosity in the superfluid phase, weak interaction, very efficient cooling, and the Transparency for electromagnetic radiation up to wavelengths in vacuum ultraviolets (VUV) Spectral range \citep{enss_low-temperature_2005}. Helium has therefore in the complete visible spectrum no transitions from the ground state. Through the noble gas configuration, helium has a spherically symmetrical electron distribution \cite{lewis_helium_2014}, it can hardly be polarized and is the least reactive of all the elements.




\subsubsection{Helium Droplets}

The production He droplets had to overcome first one principal problem, its liquefaction. At the end  of 19th century many gases were liquefied for the first time by applying pressure at room temperature. However, for some gases, such as He and hydrogen, this method was not successful. In 1922 Kamerlingh Onnes reached temperatures below $1K$ by reducing the vapor pressure above liquid helium to about $2*10^{-5}$ bar with a series of pumps \cite{van_delft_discovery_2010}. the Joule–Thomson effect \cite{weinberger_discovery_2013} is in this case the responsible for Onnes experiment to reach this low temperatures. The basic idea is that under suitable conditions a gas in expanding performs work against its internal forces. basically the gas is expanded through a small nozzle thermally isolated from its surroundings. The expansion under theses conditions takes place at constant enthalpy, since the expansion nozzle performs no work. following the next relation:
\begin{equation}

W= H_{1}-H_{2} = (U_{1}+p_{1}V_{1})-(U_{2}+p_{2}V_{2})
\end{equation}
where H is the entalpybefor and after, $U=\dfrac{3}{2}Nk_{b}T$ for ideal gases and $pV=Nk_{b}T$ \cite{enss_low-temperature_2005}. Under Joule–Thomson effect conditions, $W=0$ so $H_{1}=H_{2}$, this expansion leads to a cooling or a warming and under certain conditions, becomes supersaturated. As a result, condensation takes place and a beam of clusters is formed.

Helium nanodroplets are typically produced by a continuous or pulsed adiabatic Expansion of pre-cooled helium through a small aperture from a reservoir into a vacuum  \cite{stienkemeier_spectroscopy_2006}. In this process a droplet jet is formed, and its characteristics (blasting speeds and size distribution) can be changes due the manipulation of the setup. For example, $\bigtriangleup$pressure between the reservoir  (usually in the range of a few to 10MPa) and the vacum chamber, the nozzel temperatures(from a few K to $T \leqslant 40K$), or the nozzel size (with pinholes of diameter rounding  $5-20 \mum$)

Theres is no mathematical aproache of the physics behing this cooling expancion that relates this properties but Usually,
Raleigh scattering measurements in combination with an empirical scaling law \cite{hagena_cluster_1972} are used to estimate the mean cluster size giving a certain degree of control over the cluster size distribution by adjusting the nozzle width and the source pressure. The droplet size distribution during supersonic expansion in the follows a log-normal distribution of the form \cite{harms_density_1998}.

\begin{equation}
p(N) = \frac{1}{\sqrt{2\pi}N \sigma} \exp  \left[- \frac{(ln(N/N_{0})^2}{2\sigma^2} \right]
\end{equation}
Where \textit{N} is the number of atom in the cluster, $\sigma$ is the distribution width and \textit{$N_{0}$} is the most likely numbers of atoms.
\begin{align}
\bar N = \exp  \left(\mu+\frac{\sigma^2}{2} \right)
\end{align}

\begin{align}
\sigma N_{\frac{1}{2}} = \exp \left( \mu - \sigma ^2 + \sigma \sqrt{2 ln(2)} \right) - \exp \left(  \mu - \sigma ^2 - \sigma \sqrt{2 ln(2)}  \right)
\end{align}

Clusters produced in this way have a wide size distribution, which can be
approximated by a log-normal function (Granqvist and Buhrman, 1976) 
The expansion conditions can be divided into three modes, which are
differ as to whether the medium to be expanded differs from the medium to be expanded before expansion.
is in the gas phase or the liquid phase. The three regimes are shown in Fig.
2.1 through excellent regions and trajectories in the phase diagram for helium
In regime I the expansion is called subcritical. In regime I, the expansion is called subcritical, begins in
of the gas phase and leads to droplet formation via condensation. Regime II
includes trajectories which begin near the critical point and thus to
uncontrollable fluctuations during expansion. Regime III is
also called supercritical expansion which starts already in the liquid phase
and by breaking up the liquid drops form.
