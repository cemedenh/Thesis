
\documentclass[a4paper,12pt,bibtotocnumbered, twosite]{scrreprt}

%scrreprt

\usepackage[utf8]{inputenc}
\usepackage[T1]{fontenc}
\usepackage{amsmath}
\usepackage{graphicx}
\usepackage{placeins}
\usepackage{float}
\usepackage{array}
\usepackage{booktabs}
\usepackage{wrapfig}
\usepackage{cite}
\usepackage{etoolbox}
\usepackage{enumitem}
\usepackage{siunitx} 
\usepackage[hyphens]{url}
\usepackage[english]{babel}
\usepackage{acronym}
\usepackage{braket}
\usepackage{sidecap}
\usepackage{booktabs,array}
\newcolumntype{N}{>{\bfseries\footnotesize}l}
\usepackage[table,xcdraw]{xcolor}
\usepackage{caption}
\usepackage{subcaption}
\usepackage{grffile}
\usepackage{xcolor}
\usepackage{framed, color}
\usepackage{empheq}
\usepackage{fancyhdr}
\usepackage{amssymb}
\usepackage{floatflt}
\usepackage{slashbox}

%\usepackage{tikz}
\newcommand{\circled}[2][]{
            \tikz[baseline=(char.base)]{
            \node[shape=circle,draw,inner sep=0.2pt] 
            (char) {\phantom{\ifblank{#1}{#2}{#1}}};
           \node at (char.center) {\makebox[0pt][c]{#2}};}}
\robustify{\circled}
            

\setcounter{tocdepth}{2} 
\setcounter{secnumdepth}{2} 
\usepackage{geometry}
\geometry{a4paper,left=30mm,right=30mm, top=3cm, bottom=3cm} 

% Spezialpakete
\usepackage{epigraph}
\setlength{\epigraphrule}{0pt} % kein Trennstrich


\usepackage[onehalfspacing]{setspace}
\usepackage[pdfborderstyle={/S/U/W 1}]{hyperref}


\newcommand{\titel}{Single Shot Electron Imaging of He and Ne Clusters Under Strong Mid-Infrared Laser Fields.}
\newcommand{\welchethesis}{Master of Science in Applied Physics}
\newcommand{\thesisofwas}{A thesis submitted in partial fulfillment of the
requirements for the degree in }
\newcommand{\welchesInstitute}{Institute of Physics}
\newcommand{\welcheUni}{Albert-Ludwigs-Universität Freiburg}



\newcommand{\kurztitel}{Template Abschlussarbeit}
\newcommand{\autor}{Cristian Enrique Medina Hernandez}
\newcommand{\datee}{12. August 2019} % Abgabedatum
\newcommand{\place}{Freiburg}
\newcommand{\referent}{ Prof.\ Dr.\ Marcel\ Mudrich and Prof.\ Dr.\ Frank\ Stienkemeier  } 



%------------------------------------

\begin{document}


\begin{titlepage}
\centering
\includegraphics[width=7 cm]{logo}

\begin{center}    
    {\LARGE .} \\[0.5cm]
    {\large .} \\[5mm]
    {\large  \thesisofwas\ \welchethesis} \\ \welcheUni \\ \welchesInstitute  \\[5mm]
    \rule{\textwidth}{2pt}\\[0.5cm] 
    {\begin{spacing}{1.15} \huge \bfseries \titel \\
    \end{spacing}}
    \rule{\textwidth}{2pt}    
    \vfill 
 
     



    \begin{tabular}{ll} 
      Presented by & \autor \\
      Date & \datee \\
      Referent & \referent \\
    \end{tabular}  

\end{center}
    
\end{titlepage}
%---------------------------------------------------------------------


\chapter*{Abstract}

This work presents the measurements and comparison of different noble gases under mid and near-infrared fentosecond laser pulses doped with xenon (Xe), argon (Ar), calcium (Ca) and water (H$_{2}$O). A new data acquisition method to correlate single shot velocity map imaging (VMI) and time of flight (TOF) measurements on doped helium and neon clusters is introduce. Until now, kinetic energy of electrons resulting from nanoplasma explosion has been used in a statistical way \cite{stienkemeier_spectroscopy_2006}\cite{kelbg_comparison_2018}\cite{kelbg_auger_2019}\cite{kelbg_auger_2019}. With this method, the analysis of electron spectra for single Coulomb explosion of He and Ne clusters is possible.

The kinetic energy distribution showed a strong dependency on the number of electrons produced in a single explosion. For certain conditions, the process can be modeled as a uniform charged spherical cloud where the electron on the surface of the sphere will determine the maximal kinetic energy resulting from the explosion. 

It was demonstrated that there is a laser intensity threshold necessary to start an optimal ionization of the cluster. Additionally, the efficiency of the plasma formation has a strong relation to the cluster size and the doping level. Bigger droplets present a proficient plasma heating compare to small ones. Also, there exists an optimum doping level to efficiently start the plasma formation and the combination of certain dopant species appears to improve the plasma formation compare to one single dopant element. Using Xe and Ca doping, the replacement of a few Xe atom for Ca improves the ignition probability by a factor of two in contrast to the ignition using just Xe. 

%The laser pulse duration in contrast to the laser intensity, shows an important role in the ignition of the smaller droplets when interacts with the plasma resonance, increasing the signal counts autonomously the power,  concluding that the interaction time between the ionized electrons at the beginning of the process and the pulse is fundamental to the Coulomb explosion.

Finally, it is shown that the number of cycles in the laser pulse plays a crucial role in the plasma ignition. At constant pulse energy, a longer pulse is preferred despite the losses in peak intensity, because the interaction of the laser and the ionized electrons is enhance.




\newpage

\tableofcontents
%%%%%%%%%%%%%%%%%%%%%%%%%%%%%%%%%%%%%%%%%%%%%%%
\chapter*{List of abbreviations}
\begin{acronym}[DARAM]

 \acro{ATI}{above threshold ionization}
 \acro{BSI}{barrier suppression ionization}
 \acro{CCD}{Charge-coupled Device}
 \acro{CPA}{Chirp Pulse Amplification}
 \acro{CWL}{Central wavelength}
 \acro{EM}{Electro Mechanics}
 \acro{LASER}{Light Amplification by Stimulated Emission of Radiation}
 \acro{LT}{Langmuir-Taylor}
 \acro{MCP}{Micro Channel Plate}
 \acro{NIR}{Near Infrared}
 \acro{pBASEX}{polar Basis Set Expansion}
 \acro{PID}{Proportional – Integral – Derivative}
 \acro{TBR}{Three Body Recombination}
 \acro{TOF}{Time of flight}
 \acro{VMI}{Velocity Map Imaging}
 \acro{VUV}{vacuum ultra violet}
 \acro{XUV}{Extreme ultraviolet}
  
\end{acronym}
%%%%%%%%%%%%%%%%%%%%%%%%%%%%%%%%%%%%%%%%%%%%%%%
\cleardoublepage\pdfbookmark{\contentsname}{toc}

%%%%%%%%%%%%%%%%%%%%%%%%%%%%%%%%%%%%%%%%%%%%%%%


\section{Introduction}

Physicists have always wonder to explain and resolve dynamic processes in short scale times, so initial conditions of processes can be  describe in a time  evolution scale. Describe any system like this requires to acquire data in shorter windows of time, for example a film is only a consecutive sequence of  photographs that recreate a large time laps in a smaller time scale pics. For  atomic physics, we are talking about a micro-cosmos that varies from microseconds, i.e several bodies dynamics, to  attoseconds  for atoms,where time scales can go down to $10^{-9}$ $s$, requiring to create measurement methods capable to record in shorter time, while the experiment have to be done in a controllable way to ensures its reproductivility, as any scientific method.

The time window of dynamics of a sytem is related to  quantum dynamics, in a simple view also to its size. For dynamics happening in a molecule or a many body system interaction, the time window can oscillate between  microseconds to fentoseconds, although for millielectronvolt-scale $(meV)$ energy spacing of vibrational energy levels implies that molecular vibrations occur on a time scale of tens to hundreds of femtoseconds. The motion of individual electrons in semiconductor nanostructures, molecular orbitals, and the inner shells of atoms occurs on progressively shorter intervals of time ranging from tens of femtoseconds to less than an attosecond. Motion within nuclei is predicted to unfold even faster, typically on a zeptosecond time scale.

To achive this higth resolution in space and  time physicist have challenged to create systems with a well controlled spatial and temporal gradient. Fortunately nowadays, laser pulses can research up to extreme non-linear optical processes, produccing single aisolated  pulses of ultra violet(UV) waves as short as 67 $as$ \cite{zhao_tailoring_2012}.  Such fast pulses open up the possibility of time resolved measurements fort short processes like electron dynamics.  However, to do this, experimental schemes must be devised that allow these new light sources to be used to perform measurements on the microcosmos. In particular, in the last few years,  many studies at atom- and molecule-clusters had been published, From mid-infre red (NIR) interaction to UV or XUV pulses, that not just lead to a broad spectra to study but also to a large range of possible applications such as the generation of  energetic electrons and ions in the keV-regime \cite{fennel_laser-driven_2010}, as well as intensive XUV and attosecond pulses \cite{stebbings_generation_2011}. Laser pulses with peak intensitiesof up to $10^{21}$ $W/cm^{2}$  are available nowadays \cite{mikaberidze_atomic_1981} comercially so the difficulty and expensive of the experiments source also are easy.

But Having this is never enough, Lasers is just one huge step in order to control and ignite atomic processe in controlled standard. The other step needes is to acquirte the information we want from this proceses. For this porpouse several thecniques are available  depending the dnature of the process. For this particular  work we are interested in two particular thecniques, Velocity map image (VMI) and Time of flight (TOF).
Since its invention, this two thecniques has become two of the most commun and important measurement techniques in high energies physics. But detecting a isgnal is just one part of the job, the new laser advances like the  generation of coherent high-intensity laser pulses with intensities up to $10^{22} W/cm^{2}$  allow multiphoton ionization that allows to get time resolved measurements. These advances have enabled the development of new research areas, as well as the investigation of ultrafast dynamics in highly excited matter to nanometer size.

In this thesis we focus our efforts  on the ionization process by NIR femtosecond pulses in doped clusters from
$He$. The interaction of the dopped He droplet with the Laser field result in a energy transfer to the droplet  that ignite a ionization process, known as a nanoplasma. This resonant interaction of the laser field with a collective oscillation of the electrons in the plasma is driven by the laser field \cite{fennel_laser-driven_2010}. This process, caused predominantly by electron impact ionization, makes an avalanche-like ionization of the atoms in the cluster, leading to a heating of the plasma and, as a result, to hydrodynamic expansion and Coulomb explosion of it. To the analisis of this proceses two tecniques were used to study the electrons as well as the iionr resulting in the coulomb explotion. A velocity map imaging and a Time of flogth technique are set up in parallel to acquire the data and reconstruct the initial energies and configuration of the plasma in study. 
In the First chapter we will present a brief introduction to short plasma interactions an a basic backgourn of coulumb ionization in order to understant the physical meaning of the reaction.
in the secont chapter a more detailed explanetion of the set up used is donde. showintg from the ceation of the He droplets proces to the detection process, going througth the dopping, and ignition process
For the thirtd chapter a detailed explanation on the correlation met´hod for the VMI-TOF meassuments is done, and showing the proces of the data acqiuisitzion and its advantages
In the fourth chapter we present the correlkated data and its analisis. finally the last chapter we present the conclusion of the experiment itself also as the data analisis and future works will needed to improve this proces as well.


I. INTRODUCTION
Ion imaging techniques in the field of molecular reaction dynamics photofragment  imaging,  photoelectron  imaging,reaction   product   imaging have   proven   to   be   of   high value.
1–3 Especially, the capability of probing the full three-dimensional velocity distribution of scattered particles under study in a single image has contributed to the importance of this method, which has the multiplexing advantage of detecting  particles ions  or  electrons of  all  velocities  the  term velocity refers to the vector quantity whereas speed denotes the scalar! simultaneously. Because the detection of particles often  involves  multi-photon  ionization MPI schemes  the ion imaging technique is widely applicable, and is shown to compare  well  with  established  one-dimensional 1D time- of-flight TOF 4–6 and  Doppler  methods.7,8
As  in  conventional  TOF/MPI  techniques,  the  images  can  often  be  obtained  in  a  mass  and  internal state  selective  way  and,  in addition, they can provide information on orientational and alignment effects. 10
In  contrast  to  the  conventional  time-of-flight  method, where kinetic energy release information is contained in the temporal structure in the arrival period of electrons or ions of a specific mass, the ion imaging technique extracts all information kinetic  energy  and  angular  distributions from  the spatial appearance of the two-dimensional~ 2D! image. From an image the full three-dimensional 3D information can be reconstructed  by  means  of  an  Abel  inversion or  back projection method.11
This  also  implies  that  the  kinetic  energy resolution attainable is ultimately limited by the quality of  spatial  mapping  by  the  detection  system.  In  some  cases imaging appears less favorable compared to TOF methods in this respect.
In order to exploit the imaging method to its full poten-
tial  one  needs  to  explore  methods  for  improvement  of  the
spatial  quality  of  the  2D  image.  Since  the  mapping  of  3D
distributions of charged particles onto the 2D detector is par-
ticularly  dependent  on  the  electrode  configuration  used  to
form the extracting electric field, this study is concerned with
the  comparison  of  conventional  grid  electrodes  versus  the
application  of  a  simple  three-plate  electrostatic  lens  with
open electrodes, which can be classified as an ‘‘asymmetric
immersion lens.’’ It shall be pointed out how this lens avoids
distortions  commonly  present  in  imaging  with  grids,  along
with  having  additional  appealing  features.  It  turns  out  that
the ion lens can be operated such that particles with the same
initial velocity vector are mapped on the same point on the
detector,  irrespective  of  their  initial  distance  from  the  ion
lens axis. A more accurate description for the imaging tech-
nique using electrostatic lenses is therefore
velocity map im-
aging

\newpage  
% !TeX spellcheck = en_GB
% !TeX spellcheck = en_US 
\chapter{Theoretical Background}

This chapter presents the theoretical background necessary for the development of this project. First we present the background for cluster formation in He and Ne, followed by the basic theory of single atom ionization, cluster ionization, plasma formation and the Coulomb explosion process. Finally an analytical model for a uniform charge ionic cloud is formulated.

\section{Helium Nanodroplets}

The combination of cryogenic matrix isolation, discovered in 1954 \cite{whittle_matrix_1954}, and the now well-defined properties of helium (He) \textit{Kapitza et al} \cite{kapitza_viscosity_1938}, leaded to the creation of an excellent molecular matrix like the helium nanodroplets\cite{stienkemeier_spectroscopy_2006}.
Helium has unique properties that make it a perfect source for the nanophysics experiments. For example, it has any optical transitions in the entire infrared and visible regime\cite{atkins_liquid_2014}. In addition, helium clusters are able to pick up atoms and molecules. It creates different complexes of the species embedded in the interior acting as an ideal matrix for spectroscopy of atoms and molecules. \cite{stienkemeier_spectroscopy_2006}\cite{toennies_superfluid_2004}.

The size of a Helium cluster can reach up to $10^{8}$ atoms at ultra-cold temperature  (close to 0.37 K) \cite{toennies_spectroscopy_1998}\cite{enss_low-temperature_2005}.
Two main advantages of this cooling properties arise. First, dopants in the helium nanodroplet are set to their absolute vibronic ground states, avoiding other possible spectra and establishing the cluster in a specific state. Second, the fast cooling helps in the formation of isomers that are difficult or impossible to generate with other methods \cite{nauta_nonequilibrium_1999}. Third, because the superfluid phase of helium \cite{grebenev_superfluidity_1998}, the bond between dopants and helium is weak. In contrast to spectroscopy in other matrices with higher temperatures, the optical transitions of many dopants are barely influenced by the helium \cite{toennies_superfluid_2004}. 
The theory of He superfluidity is not part of this section, this information is well documented in other sources. We reference \textit{ Enss et al} work \cite{enss_low-temperature_2005} where all theory is presented to the reader. In the next section we will dedicate a bigger effort to explain the theoretical and technical background of the helium nanodroplets formation as well as their physical properties.

\subsection{General Properties of Helium}


At room temperature, helium is a light inert gas. It is odorless, colorless, tasteless, and after hydrogen, the second most abundant element in the universe.  \cite{enss_low-temperature_2005}. It has a simple 2 atoms structure, exhibiting numerous exotic phenomena whose theoretical descriptions are rather complex in some cases, i.e as a quantum fluid. Helium have two stable isotopes $^{3}He$ and $^{4}He$.  $^{4}He$ has two electrons, two protons and two neutrons, no nuclear spin and no total spin, pertaining to the bosonic family, while $^{3}He$ with only one neutron has a spin of $I = 1/2$ and belongs to the fermions \cite{atkins_liquid_2014}.

The bosonic state $^{4}He$ is especially of interest, at temperature T$\leqslant$2.8K and under normal pressure has a phase transition from "normal liquid" $He-I$ to super liquid $He-II$ \cite{swenson_liquid-solid_1950}, in which the helium can be described as a Bose-Einstein condensate. Even the fermionic $^{3}He$ exhibits this phase transition at T$\leqslant 0.03K$ \cite{halperin_properties_1978}.

The superfluidity of He-II, at temperatures close to absolute zero, brings with it some unique features. The essential properties for this include an almost disappearing viscosity in the superfluid phase, weak interaction, very efficient cooling, and the transparency for electromagnetic radiation up to wavelengths in vacuum ultraviolet (VUV) spectral range \cite{enss_low-temperature_2005}. In the complete visible spectrum He havs none transitions from the ground state. It has a noble gas configuration and  a spherically symmetrical electron distribution, making He hardly to be polarized and  react with other elements  \cite{lewis_Helium_2014}.

\begin{figure}[h!]
\centering
\includegraphics[width=8 cm]{../Images/He_temp_phases.png}
\caption[Helium phase diagram]{Helium Phase Diagram. $^{4}He$ phase transition from liquid to superfluid, also known as Helium-II. At $2.5$ MPa the critical temperatures is $2.17$ K. Taken from \cite{noauthor_Helium_nodate}}

\end{figure}


\subsection{Formation and Properties of He Droplets.}

At the end  of the 19th century, many noble gases were liquefied for the first time by applying pressure at room temperature. However, for helium and hydrogen, this method was not successful. In 1922 Kamerlingh Onnes using the Joule–Thomson effect \cite{weinberger_discovery_2013} reached temperatures below $1$ K in liquid helium by reducing the vapor pressure to about $2*10^{-5}$ bar with a series of differential pumps \cite{van_delft_discovery_2010}. Under suitable conditions an expanding gas can performs work against its internal forces. When a gas is expanded through a small nozzle thermally isolated from its surroundings, the expansion is done at constant enthalpy due the expansion nozzle performs none work. It follows the next relation.

\begin{equation}
W= H_{1}-H_{2} = (U_{1}+p_{1}V_{1})-(U_{2}+p_{2}V_{2})
\end{equation}

Where H is the enthalpy before and after, $U=\dfrac{3}{2}Nk_{b}T$ is the internal energy, and follows the  ideal gases law $pV=Nk_{b}T$ \cite{enss_low-temperature_2005}. Under Joule–Thomson effect conditions, $W=0$ so $H_{1}=H_{2}$, the gas become supersaturated. As a result, condensation takes place and a He cluster is formed.

Helium nanodroplets are typically produced by an adiabatic expansion of pre cooled gas through a small aperture from a reservoir into a vacuum. In this process a droplet jet is formed, and its characteristics (blasting speeds and size distribution) can be changed with the manipulation of the setup \cite{stienkemeier_spectroscopy_2006}. For example, different pressures on the reservoir and the vacuum chamber (usually in the range of a few to $10$ MPa), the nozzle temperature (from a few K to $T \leqslant 40$ K) or the nozzle size (with pinholes of diameter rounding 5-20 $\mu$m) \cite{schomas_compact_2017}.

\begin{figure}[h!]
\centering
	\includegraphics[width=0.6\textwidth]{../Images/jet_scketch.png}
	\caption[Scheme for a nozzle expansion]{ a) Schematic representation of the processes leading to the formation and subsequent cooling of helium droplets in a gas expansion. b) Calculated dependence of the droplet temperature on time for $^{4}$He and $^{3}$He droplets after they have left the cluster. Taken from \cite{toennies_superfluid_2004}	}
	\label{img:jet}	
\end{figure}

\begin{figure}[h!]
\centering
	\includegraphics[width=0.5\textwidth]{../Images/waanderwaal_hehe.PNG}
	\caption[Waan der Wall He-He potential]{ Van der Waals potential for He-He interaction. Taken from \cite{blanco_quantum_2010}}
	\label{img:WanderHe}
\end{figure}

When helium expands after the nozzle, its potential energy alters to kinetic energy in a supersonic flow field. After the expansion into the vacuum, the gas becomes supersaturated, creating a beam cluster. These clusters are made of atoms or molecules, held together by Van der Waals forces (He-He interaction), that share the same kinetic vector. When two particles travel as close and parallel to each other, bonding is possible as shown in Fig \ref{img:WanderHe}. From the re
ference frame of the He cluster, each of the molecules is close to zero movements and superfluidity is achieved \cite{hagena_cluster_1972}.
 
Depending on the buffer gas used, the mechanisms for cluster formation in the supersonic expansion is as a condensation either from the gas phase or the liquid phase. Fig 	\ref{fig:tsHe} is a phase diagram He, Ar and H$_{2}$ at different pressures ($p'=P/P_{critical},$). An isentropic expansion (adiabatic and reversible) is represented by a vertical line. Clusters formed by condensation from the gas phase occur when the expansion crosses into the two-phase region on the right-hand side of the critical point. Clusters formed by fragmentation of the liquid phase occur when the expansion crosses into the two-phase region on the left-hand side of the critical point\cite{knuth_average_1999}. The curves represent the regions where the supersonic expansion is possible and the temperatures that each gas should have in order to achieve clustering and cooling \cite{knuth_average_1999}.

\begin{figure}[h!]
	\centering
	\includegraphics[width= 8 cm]{../Images/dimensiones isentropic diagram.png}
	\caption[helium isentropic diagram]{Dimensionless phase diagram for He, H2 and Ar. Where T is dimensionless $T'=(T -T_{tp})/(T_{cr}-T_{tp})$, same as entropy $S'=(S-S_{cr})/\Delta S_{tp}$ and $x$ is the fraction of the fluid in the gaseous phase, where the subscripts $cr$ and $tp$ refer to the critical point and triple point respectively, and $\Delta S$ is the entropy change for vaporization. The curves are drawn as guides to the eye, not exact measurements, taken from \cite{knuth_average_1999}.}
	\label{fig:tsHe} 
\end{figure}
  

There is no mathematical approach of the physics behind this supersonic expansion. Assuming a certain degree of control over the cluster size distribution by adjusting the nozzle width and the source pressure. Raleigh scattering measurements in combination with an empirical scaling law \cite{hagena_cluster_1972} can be used to estimate the mean cluster size. The droplet size distribution during supersonic expansion in the follows a log-normal distribution of the form \cite{harms_density_1998}.

\begin{equation}
p(N) = \frac{1}{\sqrt{2\pi}N \sigma} \exp  \left[- \frac{(ln(N/N_{0})^2}{2\sigma^2} \right]
\end{equation}

Where \textit{N} is the number of atom in the cluster, $\sigma$ is the distribution width and \textit{$N_{0}$} is the most likely numbers of atoms. Following it give a mean value.

\begin{align}
\bar N = \exp  \left(\mu+\frac{\sigma^2}{2} \right)
\end{align}

With a half width maxima of \cite{harms_density_1998}

\begin{align}
\sigma N_{\frac{1}{2}} = \exp \left( \mu - \sigma ^2 + \sigma \sqrt{2 ln(2)} \right) - \exp \left(  \mu - \sigma ^2 - \sigma \sqrt{2 ln(2)}  \right)
\end{align}

\begin{figure}[h!]
\centering
\includegraphics[width= 10cm]{../Images/expansion_regimes.PNG}
\caption[Phase diagram for Expansion regimens]{Expansion regimes. $^{4}$He Pressure-Temperature phase diagram for Nozzle beam expansions starting at a backing pressure  of 20 bar and $a$ temperature. As discussed, qualitatively different behaviors are shown for the regime I-II and II where starting in the gas phase,  near the phase transition respectively. Taken from \cite{buchenau_mass_1990}. }
\label{fig:ExpRegim}
\end{figure}

As show in Figure \ref{fig:ExpRegim}, the initial gas conditions (pressure, temperature and nozzle size) in the free expansion phase will determine the characteristics of the final helium beam. From here, three main regimes can be define.

Regime I or sub-critical expansion, begins in the gas phase and leads to droplet formation via condensation. This is the case of most expansions since the pressure is located below the critical pressure $P_{c}$.
Regime II, also called as critical expansion, is a long-winded regime that includes all trajectories which are near the critical point, leading to random expansion and difficult control of the beam due the large fluctuations in density.
Regime III, the super-critical expansion, starts at low temperatures where the helium stops behaving as an ideal gas, expecting flashing or cavitation  breaking up the liquid drops jet. \cite{buchenau_mass_1990}

super-critical and sub-critical regimes have been studied  in the last several years and  are clearly identified in the resulting size distributions. Figure \ref{img:dropletSize} shows that super-critical expansion forms large droplets (usually between $20-100$ nm diameter) while a sub-critical expansion is suited to generate small droplets (around $5-10$ nm).  A simple relation that can be done to calculate the size or number of atoms in a Custer is using. 

\begin{equation}
r=N_{1/3} * \rho A
\end{equation}

Where $r$ is the radius of the beam, and $\rho$ its density, in this case, for helium $\rho =0.0022$ A$°$  \cite{stringari_systematics_1987}, but this approximation is not exact due the variation in He density at this temperature. As expected in both regimens, for creating larger helium nano droplets, higher He pressure and lower nozzle temperature are used. For our experiment a $5 \mu$m nozzle was used at temperatures oscillating between $10-20$ K, with a backing pressures of $30, 45 $ and $ 50$ bar.

\begin{figure}[h!]
\centering
\label{img:dropletSize}
\includegraphics[scale=0.4]{../Images/sizes_regimen.PNG}
\caption[Expansion droplets Regimens]{Sizes of the $^{4}$He droplets  as a function of nozzle temperature T and  pressures, based on \cite{toennies_spectroscopy_1998}, using a $5 \mu $m nozzle. The sub and super critical regimes are clearly differentiated. Taken from \cite{stienkemeier_spectroscopy_2006}}
\end{figure}

\section{Neon Clusters}

Neon (Ne) is the second lightest inert gas with atomic number 10. It has 3 stable isotopes in nature, the $^{20}$Ne with more than 90$\%$ of abundance, followed by $^{21}$Ne and  $^{22}$Ne \cite{meija_atomic_2016}. At extreme temperature, Ne is solid as shown in the graphic \ref{fig:Nephases} and its triple point is around $T_{p}=24$ K \cite{young_phase_nodate}. Neon is a noble gas, it shares most of the properties already mentioned from helium, except for its superfluidity.  It has a a quite large ionization potential for its first electron at $Ip=21.56$ eV \cite{iablonskyi_slow_2016}, what makes quite suitable to use it as a matrix in strong laser fileds, because at low intensities it does not interact with the light source and dopants can be carried out in a non-interactive way \cite{campargue_atomic_2012}.
  

\begin{figure}[h!]
\centering
\begin{subfigure}[l]{1\textwidth}
\includegraphics[width=0.45\textwidth]{../Images/Ne_temp_phases.png} \hfill
\includegraphics[width=0.4\textwidth]{../Images/T-s ne phase diagran.png}
\end{subfigure}
\caption[Neon phase-Isentropic diagrams]{On the left, Neon phase diagram. taken from \cite{young_phase_nodate}, on the right, $T-S$ phase diagram of Ne. The critical point is located at $T_{c}= 44.49$ K and a molar entropy of $S_{c}=30.76$ J/(mol K). The dashed lines represent regions. Taken from \cite{christen_supersonic_2010-1} }
\label{fig:Nephases}
\end{figure}

Ne cluster has been proved to provide an ideal medium for chemical reactions as solvation effect and heterogeneous chemistry at a microscopic level \cite{gough_infrared_1985}. With a regulated doping process reactants are deposited in a controlled way in the cluster and it becomes a nanoreactor equivalent \cite{gaveau_reaction_2001}.

Several studies have been realized on the characterization of Ne clusters, for example  by \textit{R. Von Pietrowski et al}\cite{von_pietrowski_fluorescence_1997} who studied the electronic excitations of Xe atoms and Xe$_{2}$ molecules  embedded in free Ne clusters. On contrary to helium, is important to work with neon at temperatures and pressures far from its solidification point. At extreme low temperatures, small differences in pressure leads to big size changes on the clusters, the higher the pressure in the nozzle the bigger the droplets. As an example, in \textit{Pietrowski} work, it was shown that small droplets, $N=300$, where $N$ is the number of atoms in the cluster, are in a "liquid" state but for bigger droplets solidification starts to be present. 

%In addition, the location of the dopant will be affected drastically by these sizes changes. When the droplet is in a liquid state the dopant atoms are free to move to the center contrary to denser droplets, where the dopant will stay at the surface.


The T-S representation of Ne, Fig \ref{fig:Nephases} shows the isentropic processes as simple vertical trajectories. The color plot makes visible the two-phase region where condensation may take place. The dashed lines represent isobar lines at p$= 100,1000,1000,10000,$ and $100000$ Pa from left to right respectively. On one hand, supersonic expansions which originate in the vapor phase (yellow), at a very low source pressure equivalent to a comparatively large stagnation entropy $s_{0}$, will not reach the condensation region (green). On the other hand, for negative entropies the solid state (red) is always reached although for lower temperatures and a relative small entropy, the liquid state is the predominant (blue).\cite{christen_supersonic_2010-1} 

\section{Composite Clusters}

A composite cluster or doped cluster is an atomic agglomeration that contains two or more different atomic elements. The main properties in non-doped clusters (with a single species) are usually set as a function of its size but for doped clusters, the interaction between the elements creates new degrees of freedom that makes its properties more complex. For example, the new mixture will have different structural interactions due the spatial distribution of the species \cite{stienkemeier_spectroscopy_2006}. Hence, composite clusters exhibit a more diverse behaviour and offer more opportunities to study different characteristics of the material.

There exist two main technique for the creation of composite clusters. A co-expansion of a previous mixed gas \cite{tchaplyguine_variable_2004} and a gather process where the first cluster is crossed with an atomic beam of the doping species.

In the co-expansion technique, clusters are produced by adiabatic supersonic expansion of mixed gas through a narrow conical nozzle into vacuum\cite{tchaplyguine_variable_2004}. The resultant properties of the binary cluster are less well known because it involves several unknown parameters as interactions between the elements, the condensation ranges of the bulks and even the affinity of the materials\cite{reinhard_introduction_2004}.

The second technique used in this study, is the one called pick-up process \cite{gough_infrared_1985}. The idea is simple, as a snowball on its way downhill collects or pick-up more snow, the cluster after being directional selected by a skimmer, crosses a doping cell full of a dopant gas at low densities ($10-2 Pa$) \cite{stienkemeier_spectroscopy_2006}. As a result, the gas atoms that are along the droplet cross sections will be captured by the beam and travel with it. The probability for a droplet to collect $k$ atoms or molecules via inelastic collisions depends on the length of the oven cell $l$, the cross section of the droplets $\sigma $, and the particle density inside the cell $n$. As $l$ and $\sigma $ remains constant, varying the density in the doping cell can regulate the abundance of $k$, following Poissonian statistics.

\begin{equation}
P_{k}(l,n,\sigma)=\dfrac{(ln\sigma)^{k}}{k!} e^{(-ln\sigma)}
\end{equation}

Two important properties of these relations can be deduced. First, the maxima of different cluster sizes  are equidistant, $n_{max}=\dfrac{k}{l\sigma}$ and second, the exponential function in the equation  becomes  nearly  one for  small  particle  densities \cite{bunermann_modeling_2011}.

Every pick-up process leads to an energy transfer to the droplets. As the dopant rapidly cools down, the energy transfer to the cluster causes an evaporation of some atoms to keep the temperature unchanged in the cluster. This evaporation or "shrinkage", leads to a decrease in the cross section of the droplet and the probability to collect further particle. At a certain energy entry, the complete droplet evaporates if to many dopants access to it. With the average kinetic energy $E_{kin}$, and $E_{in}$ the internal.  The involved energy is composed of the following contributions \cite{bunermann_modeling_2011}.

\begin{equation}
E=\langle E_{kin}\rangle + E_{in} + E_{binding} + E_{cluster}
\end{equation}

Where

\begin{equation}
\langle E_{kin}\rangle \approx \dfrac{3}{2}k_{b}T + \dfrac{1}{2} m v^{2}
\end{equation}

is the final kinetic energy of the droplet depending on it mass, velocity and temperature in the gas cell.


\begin{figure}[h!]

\centering
\includegraphics[width=14cm]{../Images/He_evaporation (2).png}
\caption[Helium creation, doping and evaporation sketch]{Animation of the Helium creation, doping and evaporation. From left to right, the helium droplet production after being released by the supersonic jet, the cluster formation, the doping process and finally the helium shrinking process.}
\label{fig:shrink}
\end{figure}

For He, several studies have focused on the $E_{binding}$ with $^{4}$He, so the energy binding  for a broad number of materials is known. It is also important to take into account that the binding energy includes the cluster-dopant binding as well as the dopant-dopant relation.\cite{toennies_spectroscopy_1998}. The energy bounding for example of Xe-He is arround $26.9 $ meV \cite{lewerenz_successive_1995}, or He-H$_{2}$O is about $0.1$ eV \cite{lewis_Helium_2014}.

\section{Cluster-Intense Fields  Interaction}


The understanding of the interaction atom-fields has been study broadly in physics since Einstein photoionization theory \cite{einstein_uber_1905}, who gave a base on all the quantum electrodynamics theory. The basic idea supporting  this theory is the behavior of light as an electromagnetic field, where the electron, as a bounded charge in the atom, can be affected.  This quantum dynamic theory is well understood since 1957 for small atoms, with one, two or few electrons \cite{a._bethe_quantum_1957}, but still big molecules and atoms have been challenging scientist for years. In this chapter we will give a brief introduction to the atomic photo ionization process, the multi-photoionization and tunneling processes, to finish with a detailed presentation of strong field interaction with clusters and the Keldish theory.

\subsection{Photoionization for Single Atoms}

The photoionization process describes the withdraw of an electron from a bound state into the continuum by interaction with electromagnetic field radiation\cite{berkowitz_photoabsorption_1979}. The atomic bounded electrons while going through an electromagnetic laser filed can absorb enough energy to get excited and fly away from the nucleus. A bound electron only can escape from an atom by absorbing photons its energy exceeds the binding energy \cite{einstein_uber_1905}. When the photon energy of the laser is smaller than the ionization potential of the target, the electron can absorb two or more photos in the ionization process, this is called Multi photon ionization (MPI). Another possible process is called, tunneling ionization, where due the quantum mechanical properties of the electrons under certain conditions absorbs enough energy to be close to the continuum, due it quantum dynamic properties it can escape in a probabilistic state.

There are a variety of theoretical approaches to describe the interaction of laser fields with atoms. The Hamiltonian of a system of $N$ particles (ions and electrons) with pairwise Coulomb interactions under the action of an external time-dependent electric field has the form  \cite{mikaberidze_atomic_1981}

\begin{equation}  \label{eq:hamiltonian}
\centering
H = \displaystyle\sum_{1 \leqslant i \leqslant N}^{} \dfrac{P_{i}}{2m_{i}} + \displaystyle\sum_{1 \leqslant i < j \leqslant N}^{} \dfrac{q_{i}q_{j}}{\mid r_{i}-r_{j} \mid} + \displaystyle\sum_{1 \leqslant i \leqslant N}^{n} q_{i}r_{i}\varepsilon(t)
\end{equation}

where $ r_{i,  p_{i}} $ and $ q_{i} $ are the coordinates, momenta and charge of the particles, including the interaction between the classical electric field and $ \varepsilon(t) $ where 

\begin{equation}
\varepsilon(t) = \varepsilon_{0} e_{z}cos(\omega t + \varphi)
\end{equation}

The process that drives ionization can be divided on two regimes, a quantum electrical regime and a classical one \cite{karnakov_strong_2009}. Equation \ref{eq:hamiltonian} use the non-relativistic approximation and neglect contributions from magnetic fields. The classical description of the laser field is a good approximation for intense laser pulses \cite{gruner_femtosekundenspektroskopie_2013}.

An electron in the initial level with energy $E_{i}$ can absorb a photon with energy $\hbar \omega$ leading to final transition where $E_{f}-E_{i}=\hbar \omega$. When the energy of the photon is larger than the bounding energy, the electron is free with a remaining kinetic energy $E_{kin} = \hbar \omega - I_{pot}$ \cite{becker_vuv_1996}.
In classical mechanics the probability of the energy transition depends directly on the cross section $(\sigma)$ of the electron and the field. However, in quantum mechanics, the photoionization cross section is related to its transition probability between the initial and the final state given by Fermi’s golden rule \cite{fermi_quantum_1932}
 \begin{equation} 
 \label{eq:transitionprobability}
W_{|i\rangle \rightarrow |f\rangle} = \frac{2\pi}{\hbar\hbar} |\langle f|H|i\rangle|^{2} \delta(E_{i} - E_{f}-\hbar\omega)
 \end{equation}
 
 \begin{equation}
 \label{eq:crosssecQ}
 \sigma(\hbar \omega) = \frac{2\pi}{3} \alpha a_{0}^{2} \hbar \omega |\langle f|r_{n}|i\rangle|^{2}
 \end{equation}
 
When Eq. \ref{eq:transitionprobability} is the transition probability of one electron to jump from initial state $i$ to final state $f$, where $H$ is the Hamiltonian operator. Eq. \ref{eq:crosssecQ} is the consequent cross section considering only the dipole part of the interaction Hamiltonian, where $\alpha$ is the fine structure coefficient, $r_{n}$ is the position operator of the electron $n$ \cite{fermi_quantum_1932}.

The energy photon needed to ionize an atom, is directly proportional to the energetic distance between the electronic states and the ionization threshold. For states closer to the ionization potential a UV photon can be enough to free an electron but for inner electrons higher photon energies are required, varying from few eV to the order of several of keV, needing radiation sources at shorter wavelengths such as XUV to X-rays.\cite{becker_vuv_1996}

After photoionization is complete, the electronic structure of the atom needs to rearrange due to the vacancy left by the ejected electron. Relaxation processes can happen during this time. An electron from the outer shell will decay and replace the freed one, therefore the energy difference of the needs to be released in the form of a fluorescence photon or Auger electron. In case of a fluorescence decay, the ionic state of the target does not change since no additional electron is released. When there is an excess of energy  a second electron from the outter shell can be released, this proces is called Auger decay \cite{rafipoor_two-color_2017}.

%\begin{figure}[h!] 
%
%\centering
%\includegraphics[width=12 cm]{../Images/text6418.png}
%\caption[Relaxation processes for photoionization]{Two example on the relaxation processes. On the left, A photon ionized an electron and the Electron $E_{in}$ replaced, expelling a fluorescent photon in the process. On the right, the energy released by the replacement electron is enough to make another electron in the outer shell to also go to the continuum, Auger electron. Taken form \cite{rafipoor_two-color_2017}}
%\label{fig:augerfluorec}
%\end{figure} 

%In example. As shown in fig \ref{fig:augerfluorec}, if a photon  with energy $\hbar\omega > E_{bin}$  ionized an electron, this will leave the atom lifting a gap. An electron in the higher levels will replace the outer one, leaving an excess of energy. The outcome will be a fluorescence process with $E_{flu} = E_{in}- E_{out}$ or , the Auger $e-$, if $ E_{in}-E_{out} > E_{bond}$ and this electron can also escape the atomic Coulomb potential \cite{schmidt_electron_1997}.



\subsection{Multiphoton and Tunnelling Ionization}

For laser fields with intensities lower than $I \leq 10^{14}$ W/cm$^{2}$ are not strong enough to change the binding potential of an atom significantly \cite{rhodes_multiphoton_1985}. Even if the photon energy is lower than the binding potential multiphoton ionization can takes place (MPI).  MPI is the simultaneous absorption of several photons to overcome the ionization barrier. The way MPI occurs depends on the laser frequency and intensity. When the intensity is much lower than the characteristic atomic resonance, MPI occurs via transitions through virtual states. Ionization by several photons at low laser intensities can be realized by the so-called resonance enhanced multiphoton ionization (REMPI) \cite{mainfray_multiphoton_nodate}.  Ionization by a REMPI process takes place in two steps. First, a resonant excitation by one or more photons occur on an electron state of the atom. In the second step, this electron state is transformed into a virtual state, to an upper state until the electron is excited by spontaneous decay. So for example, the total energy absorbed by an election until it gets ionizes is $n * \hbar\omega > I_{pot}$ where $n$ is the number of photons absorbed until it actually have enough energy to overcome the potential $I_{pot}$ \cite{gruner_femtosekundenspektroskopie_2013}.

For Laser intensities $I > 10^{14}$ W/cm$^{2}$,  with higher intensities and lower frequencies, barrier supression (BSI) and tunneling ionization (TI) is more likely to occur.  In this case, the binding potential of the atomic state is strongly affected by the electric field of the laser. Around the peak of the electric field the  potential gets narrower, and the electron in the outer states gets closer to the bidding barrier, allowing the electron to tunneling through the confining potential to the continuum  \cite{griffiths_introduction_2013}. TI is inherently a quantum process. The bending of the Coulomb potential becomes by the superposition of it and the laser field. Therefore TI must occur when the time of the ionization is shorter than a laser oscillation cycle\cite{berkowitz_photoabsorption_1979}. BSI is based on the same principle, when the laser field becomes so strong to bend the  potential that separates the highest electron level, then the electrons in this state become free electrons\cite{krishnan_doped_2011}.

In Fig. \ref{img:ionizationprocess}, we present a sketch of the 3 possible ionization processes. On the left, a simple ionization process where a photon with energy $E_{phot} = \hbar\omega$ is higher than the potential barrier. In the center, a MPI process is shown, $n$ photons excite the inner-shell electron, exiting it through a virtual level until it finally has enough energy to be free to the continuum. Finally on the left, a BSI process happens. The coulomb potential barrier is affected by the laser field bending, the outer shell electron gets closer to it until they can scape \cite{rafipoor_two-color_2017}.

\begin{figure}[h!]

\centering
\includegraphics[width = 14 cm]{../Images/photoionization2.png}
\caption[Ionization regimes]{ On the left is the sketch of a single photon ionization process, where a photon with energy $E_{phot} = \hbar\omega$ is higher than the potential barrier $I_{p}$. On the center the MPI process, inner-shell electron absorbs $n$ photons, getting excited through the electronic levels (reals or virtual) until it reaches the continuum. On the rigth,  the BSI Process, the coulomb potential barrier bends by the laser fields, been lower than the outer shell electron state, the electrons can scape. Based on \cite{rafipoor_two-color_2017}.}
\label{img:ionizationprocess}
\end{figure}


As explained the intensity of the field plays an important role in the ionization process. A rather easy way to differentiate when each process needs to be taken into account is provided by the Keldysh parameter\cite{keldysh_ionization_1965}.

\begin{equation}
\gamma_{k}=\sqrt{\dfrac{I_{p}}{2U_{p}}}
\end{equation}

Where $\gamma_{k}$ is the Keldish parameter, $I_{p}$ is the atoms ionization potential and $U_{p}$ is the ponderomotive potential defined as:

\begin{equation}
U_{p} = \dfrac{e^{2}E_{0}^{2}}{4m_{e}\omega_{0}^{2}} \propto I \lambda^{2}
\end{equation}

Where $m_{e}$ is the mass of the electron, $\omega_{0}, \lambda, I$ and $E_{0}$ are the frequency, wavelength, intensity and the peak of the electric field of the laser pulse. On one hand, when the Keldish parameter is higher, $\gamma_{k} \gg 1$ MPI regime is considered. On the other hand, the $\gamma_{k} \ll 1$ describes the TI interaction.

\subsubsection{ Keldysh Theory}

In this section we will give a brief introduction to the keldysh theory based on the work of Keldy et al \cite{keldysh_ionization_1965} and the papers review of the theory by \cite{popruzhenko_keldysh_2014} and \cite{karnakov_strong_2009}. For a deeply explanation we recommend the reader to reference this works.

The keldysh Theory, also known as the Keldysh–Faisal–Reiss theory (KFR), is well used for the description of quantum process induced by intense laser radiation. The applications and advantages of Keldysh formulation in many-body theory among  several, can overcome from, treatment of systems away from thermal equilibrium, solutions in  super symmetry methods of systems with quenched disorder or to the  calculation of the full counting statistics of a quantum observable \cite{kamenev_introduction_nodate}.

According to the Keldysh ansatz, the transition probability amplitude between an atomic bound state and the continuum by the value of the photoelectron momentum $p$ measured at the detector is given by \cite{popruzhenko_keldysh_2014}.
 \begin{equation}
 M_{k}(p) = -\dfrac{i}{\hbar} \int_{\inf}^{+\inf} \langle \Phi_{p}\mid  V_{int}(t)\mid \Phi_{0} \rangle dt
 \end{equation}

Where $M_{k}$ denotes the Keldysh transition probability, $\Phi_{0}$ is the bond state wave function unperturbed and $\Phi_{p}$ is the canonical momentum, equal to $p$, also known as the Volkov function, and $V_{int}$ is the electron field interaction operator. If the amplitude of ionization $M_{k}(p)$ is known, the differential probability to find the photoelectron in the elementary volume near the momentum $p$ is given by the momentum distribution of the photoelectrons 

\begin{equation}
dW(p)=\mid M(p)\mid^{2} d^{3}p
\end{equation}
 Giving a total probability of
 \begin{equation}
 W= \int \mid M(p)\mid^{2} d^{e}p
 \end{equation}
 
Meaning that, for enough long pulses, containing a large number of optical periods so that its electromagnetic field is close to a periodical function of time close to the initial, it is physically more appropriate to use probabilities per time unit (rates) instead of time-integrated values.

\subsubsection{Ponderomative Energy}


As soon as an electron is released into the continuum, it is under the influence of the external laser field. A description of the energy that it acquires during this interaction is given by the ponderomotive energy (PE).

\begin{equation}
U_{p} = \dfrac{e_{2}E_{a}^{2}}{4m \omega^{2}}
\end{equation}

Where $m$ and $e$ is the electron mass and charge, $E_{a}$ and $\omega_{0}$  amplitude and frequency of the electric field respectively. The formula of the ponderomotive force can be easily derived as shown in \cite{protopapas_atomic_1997}\cite{connerade_highly_1998}. Let`s consider a polarized electric field (in a.u).

\begin{equation}
E=\widehat{z}E_{a}sin(\omega_{0} t) 
\end{equation}

Considering only the $\widehat{z}$-components so we can avoid the vector sign. By classical mechanics we have.

\begin{equation}
p(t)=-\int_{t_{0}}^{t} E(t\prime)dt\prime = \dfrac{E_{0}}{\omega} (cos(\omega_{0} t)- cos(\omega_{0} t_{0}))
\end{equation}

The term on the left of the parenthesis is known as the time varying Quiver terms, and on the one on the right, reefers to the drift motion. 
Expressing the fields in terms of vector potential we will have

\begin{equation}
E(t) = \dfrac{\delta A(t)}{\delta t}
\end{equation}

\begin{equation}
p(\infty) = A(t_{0}) = -\int_{-\infty}^{t} E(t) dt = \dfrac{E_{0}}{w} cos(w_{0}t_{0})
\label{eq:pmax}
\end{equation}

in the case where the pulse  duration is big $t \longrightarrow \infty$ the $p + A(t) = 0$. This means that the momentum acquired by the electron will depend on the phase it is realized $wt$. Since the electron can be unbound in any phase of the laser pulse, will have an average kinetic energy described by

\begin{equation} 
U_{p} = \dfrac{1}{2\pi} \int (-\dfrac{E}{w} cos (wt))^{2}d(wt) = \dfrac{E^{2}_{0}}{4w^{2}_{0}} = \dfrac{p_{max}}{2}^{2}
\label{eq:pondeenergy}
\end{equation}

The pondemorotive energy also gives the maximum momentum that an electron can acquire (Eq. \ref{eq:pmax}), given at the maxima. The $\omega t$ phase relation, defines what it called \textit{the three step model} showed in Fig. \ref{fig:ponder}. The first step corresponds to $\omega t < \pi /2$ where the laser field is suppressed, and as explained above, TI or BSI can take place. The second step, is where $\omega t > 3\pi /2$, on contrary step 1 the potential barrier is enhanced, electrons in the continuum that was winning kinetic energy are caught by the potential again, being driven  back to the atom. Finally the step 3 at phase $\omega t = n* \pi$, for $n 0 1,2,3...$. At $n=2$ it is called "recollision process" of the electron. Where the electron can be caught by the potential again, and the excess of energy release another bound electrons, depending on the kinetic energy necessary \cite{krishnan_ignition_2012}

\begin{figure}[h!]
\centering
\includegraphics[width=12 cm]{../Images/ponderomotive steps.png}
\caption[Ponderomotive 3 steps]{Recollision  process at the three step model. Taken from \cite{krishnan_doped_2011}}
\label{fig:ponder}
\end{figure}


If we transform the eq. \ref{eq:pondeenergy} to laser intensities we will have $U_{p} = 9,33*10^{-14} I[W*cm^{-2}] \lambda^{2}[\mu m]$. For a MIR-pulse with intensities $\sim 10^{14}$ W/cm$^{2}$ and $\lambda \sim 3200$ nm we have electron energies between one and a hundred of eV.

\subsection{Cluster Ionization}

Cluster Ionization is a more complex process than depends mainly in the laser field interaction and atomic interactions. He clusters are mainly affected by Van der Waals attraction\cite{stienkemeier_spectroscopy_2006, the cluster are equally affected by the pulse, in other words the Field penetrates all over the cluster when it interacts with  laser fields, specifically pulsed laser with  a wavelength smaller than the cluster size.

Being a rare gas, He ionization potential is higher than many of its doping molecules used. For example, under MIR lasers helium clusters need $I > 10^{15}$ W/cm$^{2}$, so He atoms don’t get direct ionized in the beginning of the plasma generation. Other interactions need to be explained in order to describe the process properly. This section we will be based on \textit{Saalaman et. al} work \cite{saalmann_mechanisms_2006} and \textit{Grüner et. al} \cite{gruner_femtosekundenspektroskopie_2013} where the plasma formation in strong laser pulses is divided in three stages.

In the first stage called \textit{“atomic ionization”}, the doping atoms are ionized independently of each other by the electric field at the leading edge of the laser pulse, it occurs mainly through inner ionization, especially on TI or BSI. The resulting free electrons acquire positive kinetic energy and have two options, leaving the cluster or staying inside the cluster attracted by its positive ion core. After the first stage, the cluster becomes into an "ignited” nanoplasma, consisting of ions and quasi-free electrons, electrons that are free to travel inside but still into the continuum \cite{last_quasiresonance_1999}. 

The second stage is called \textit{nanoplasma expansion}. During this stage the cluster is still interacting with the laser field. The electrons and ions acquire energy becoming in a quasi-free state \cite{gruner_femtosekundenspektroskopie_2013}. Quasi-free electrons oscillates inside the cluster, driven by the laser pulse and are heated to high temperatures. The heating becomes extremely efficient when the collective oscillations of quasi-free electrons become resonant with the laser pulse, triggering a cascade reaction to more outer ionizations,  freeing the remaining electrons in the cluster, this process is called \textit{plasma resonance}\cite{saalmann_mechanisms_2006}.



After the laser pulse is over, the last stage starts. The ions continue to expand, in consequence, the radius rises as same as the cluster potential becomes smoother. So, it is easier for the highly energetic quasi-free electrons to leave the cluster, forming a coulomb explosion that destroys the cluster in a ions-electrons cascade. This process was first described by \textit{Ditmire et. al} \cite{ditmire_interaction_1996} combining high energetic collisions with cluster resonance absorption.

The graphic \ref{img:clusterpotential} shows on the left how the cluster potential is composed by the Van der Waals potential and atomic forces of the different atoms that compose the cluster. On the central images, additional to the atomic biddings, the electrical force due the ions in the cluster increase the potential, the laser pulse is still on and the quasi-free electron will gain energy while they are in this. Finally, on the right the laser field is off, the electrons have fled away and the cluster ion have been repelling each other so the potential is reduced to the minimum(just atomic interactions.)

\begin{figure}[h!] 

\centering
\includegraphics[scale=0.35]{../Images/clusterpotential.PNG}
\caption[Cluster potential regimes]{Cluster potential regimes. On the left, the atomic ionization starts the plasma formation. On the center, the quasi-free electrons auto-ionize the cluster, increasing the potential barrier and gaining energy due the laser field so a coulomb explosion can take place. On the right, The Coulomb explosion is finished, the potential is driven to it minimum and all the electrons and ions are ejected. Taken from \cite{wabnitz_multiple_2002}}.
\label{img:clusterpotential}
\end{figure}

Depending on the droplet size and the laser intensity, the Cluster can expand in two different ways. If the laser intensity is rather high and the droplet is small, a Coulomb explosion can occur. On the contrary, if the laser is not intense enough or the droplet is too big, a nanoplasma can be generated, therefore a hydrodynamic expansion will take place. 
Two forces are really important during the cluster expansion. Both act on the cluster during the phase two and three (during and after the laser pulse). The first, is the force associated with the free electrons with high kinetic energy. These hot electrons expand and pull the low energetic electrons and heavy ions on its pad \cite{ditmire_interaction_1996}. The other force acting on the cluster is due to the inner cluster charge itself. The hottest electrons in the cluster will have a mean free path large enough so they can free stream directly out of the cluster, and, if the electron’s energy is large enough to overcome the space-charge buildup on the cluster, they will leave the cluster altogether. If the charge buildup is sufficiently large, the cluster will undergo a Coulomb explosion \cite{haught_formation_1970}. According to Madison et all, a time scale for the laser pulse duration where the coulomb explosion can take place should be closer or lower to the femtosecond regime, depending on the element composing the cluster. \cite{madison_role_2004}. Based on the laser power available on modern laser pulses, the same studies present that electron after a coulomb explosion can get kinetic energy up to 6KeV.


When the intensity is not enough to make the atomic bonds to break, the electrons remain in the cluster forming a hydrodynamic expansion as a result of a conversion of electron-thermal energy to direct kinetic energy \cite{erk_nobel_2009}. The effects that the expansion has on the electron temperature can be calculated by equating the rate of change of radial kinetic energy from the thermal contribution with the rate of change of thermal energy within the cluster. When this condition is fulfilled. The electron can present a resonance condition in the cluster, traveling in the space-charged forces formed by the plasma, winning enough kinetic energy until all the system collapse.

\begin{figure}[h!]
\centering
\includegraphics[width=14cm]{../Images/cluster_regimes_2.png}
\caption{sketch of the Coulomb explosion for a doped cluster due the excitation of a pulsed laser field. At the beginning of the process the laser ionized the droplet, until some femtoseconds (up to 500fs) after, the system collapse and result into a coulomb explosion.}
\label{fig:columbexplosion}

\end{figure}


Although the two models are different in each regime, for example, at low kinetic energy or the beginning of the pulse, the Coulomb explosion produces less ions for low energies compared to the product on hydrodynamic explosions. Further, the number of high energy ion the coulomb explosion can create (although in less quantity) tent to be hotter ions too.
We have to take into account that even the two processes are described for different laser regimes. Both processes can happen in parallel, but at certain energies is clear, that one or the other will be the responsible at the end, for the collapse of the system.

\subsection{Homogeneous Charge Sphere  Model}

Once the ionization process started, an electronic cloud of quazifree electrons will be created around the remaining cluster. This is a complex physical process due its chaotic system determined by the individual velocity vectors of the particles and their interaction between the Electric field, the cluster and each other. Various theoretical approaches ranging from phenomenological models \cite{ditmire_interaction_1996} to large-scale microscopic calculations \cite{saalmann_mechanisms_2006} have been done but far for been interpreted to clusters and system with more than a few thousand of particles. Additional a  theoretical scenario of a single well-characterized cluster, irradiated by a laser pulse of a given intensity is usually far from the real experimental situation. An analytical solution is far from been formulated, we work joint to the theoretician Andreas Heidenreich to make computational simulations. This are demanding simulations that takes a large of computational power even for small cluster, the results can recreated reasonably the experimental system and so give a better background to the understanding of the plasma formation.

Another, more simple and intuitive approach is done by  \textit{Ranaul Islam et al.}, in \cite{islam_kinetic_2006} they try to express the kinetic energy distribution for an ion cloud. Two important assumptions are made for this model. First, we assume a uniformly charge distributed spherical cloud with radius $R$ and density $\rho=N/vol$ where $N$ is the number of particles inside the sphere, furthermore all the particle lays with zero kinetic energy. Second, The basic mechanism underlying the kinetic energy distribution in clusters is their Coulomb explosion, it converts the potential energy $E_{coul}$ of a partially ionized cluster atom at a distance $r$ from the cluster center into kinetic energy $E$. The probability $dP/dr$ to find an atom at a distance $4$ from the cluster center is then given by\cite{islam_kinetic_2006}

\begin{align}
\frac{dP}{dr}=\frac{3 r^2}{R^3} \Theta (R-r)
\label{density_distribution}
\end{align}

where, $\frac{dP}{dr}$ is the probability to find an electron at radius $r$ and $\Theta$ is the step function for the particles inside the radius of the sphere, with homogeneous charged density, we have $N$ particles with a charge of $q$ and just after the are ionizes they have not moved yet, then the potential Coulomb energy of an particle inside the cluster is given by

\begin{align}
E_{coul}(r)=Ne^2 \frac{r^2}{R^3}
\label{coulomb_energy}
\end{align}

for $r\leq R$. $N$ is the number of electrons in the sphere and $e$ is the elementary charge of an electron. A charged sphere like this will immediately coulomb explode and for $t \longrightarrow \infty$ all coulomb energy is converted to kinetic energy, which can be measured experimentally with the VMI. It is possible to retrieve the energy distribution out of the spatial distribution \ref{density_distribution} with \ref{coulomb_energy} as the substitution formula.

\begin{align}
dr=\frac{R^3}{2Nq^2r}dE
\label{substitution}
\end{align}

Furthermore, we define the maximum coulomb energy

\begin{align}
E(R):=E_R=Nq^2 \frac{1}{R}
\label{max_coul_energy}
\end{align}

with all this follows the energy distribution of the electrons

\begin{align}
\frac{dP}{dE}=\frac{3}{2} \sqrt{\frac{1}{E_R}} \frac{1}{E_R}\sqrt{E} \cdot \Theta (1-\frac{E}{E_R})
\end{align}

it can be seen, that the energy distribution is fully characterized by $E_R$, so it is enough to know the maximum kinetic energy, which is just the radius of the central feature in our VMI images. With this even the inverse Abel transformation can be bypassed, because the edge of a sphere in invariant for projecting the sphere on a plane.
\begin{figure}[hbtp]

\centering
\includegraphics[scale=0.4]{../Images/linemodelfit.png}
\caption[Islam Model fit]{Energy distribution for a coulomb explosion of a full sphere of electrons according to formula 3.10. The dashed line marks cutoff energy E$_{R} $}
\end{figure}


With the formula for the homogeneous density in a sphere

\begin{align}
R=(\frac{N}{\frac{4}{3} \pi \rho})^{1/3}
\end{align}

and formula \ref{max_coul_energy} the charge density in the beginning in the process can be derived to

\begin{align}
E_{max}(N)=\underbrace{\frac{e^2}{4 \pi \epsilon_0} (\frac{4}{3} \pi \rho)^{1/3}}_{=:B} N^{2/3}
\end{align}

and with this the charge density reads

\begin{align}
\rho=48 \pi^2 \frac{\epsilon_0^3}{e^6} B^3
\end{align}

In summary, the charge density can be calculated with the fit parameter $B$ or B-factor as we will named from now on.  $B$ can be retrieved by plotting the maximum kinetic energy $E_R$ as a function of the number of electrons $N$. Both can be extracted out of the VMI images, $E_R$ from the radius and $N$ from the brightness of the central feature.


\newpage
% !TeX spellcheck = en_GB
% !TeX spellcheck = en_US 


\chapter{Experimental setup}

In this chapter we will present the experimental setup used, given a special section to  the Nanodroplet generations, the doping process and the data acquisition system. A  sub-chapter is also dedicate to the single shot correlation data acquisition system tested specifically to this Master`s project.
The apparatus we worked with is part of the the  group of Molecule and Nanophysics at the University of Freiburg, Germany, and was calibrated and used for the experiments in \cite{schomas_compact_2017} and \cite{heidenreich_charging_2016}.

In figure XXX an sketch of the apparatus is shown. From left to right; The source chamber where the  ultra cold molecular beams are produced, the central chamber or "doping chamber" where the beam gas is doped via pick up process using a gas doping cell or a diffuse oven for alkaloids where gases or thermally vaporized solids and using as dopant, the last  chamber or "detection chamber" combines a VMI - TOF detection system and a  Langmuir taylor (LT) detector.
To generate the nano plasmas the apparatus was Used in two different institutions, at the Max Plank Institute for Nuclear Physics in Heidelberg and the Extreme light institute (ELI) in Szeged, Hungary, because of the special laser systems can be provided in there.

The following sections describes the essential components of the apparatus, taking special attention to the new Triggering systems implemented to correlate the VMI and TOF signals in the last part of the project.  The structure where compacted to a length of about 240 cm long, each chamber has attached its own turbo pumps with  pre-vacuum pumps and are separated by valves and skimmers. On one hand, this enables to manipulate the vacuum in an independent way and control the targets in the "detection chamber", on the other hand a allows an optimum adaptation of the suction power of the pumps to the gas load of the individual chambers as wheel as to ventilate and open, without having to disturb the entire system.

\section{Source chamber}

The Source chamber consists of a 6-way CF vacuum chamber, with a 2-stage cryostat power a cold-head that can be cooled down to 9K, located in entrance of the chamber, parallel to the floor with an attached conical nozzle for the gas expansion process. The cooling capacity of the cryostat  consists of a copper tube, into which pre-cooled helium is introduced. It can be adjusted by operating two heating resistors in combination with a sensor diode for temperature measurement and a PID controller. Controlling the resistor current the temperature at the end of the nozze can be keeping it stable. The conical nozzle used to generate the atomic beam is the standard used in the research group on Nanoplasma reasearch directed by Prf, frank Steankeirmeyer at Freiburg University.  It is made of copper and has a platinum plate on front with a hole of 5$\mu m$ of diameter for $He$ experiments and 15$\mu m$ for $Ne$ gas clusters. The diameters where choseen in order to follow the Size dependence of the \textit{Hagenas} "law",  which together with the adjustable gas pressure and the nozzle temperature regulates the flow.
The cold head-nozzle arrangement are connected to the chamber via a self-made x-y manipulator,  with a thermally insulating rubber? ring, which allows a beam adjustment in relation to the other components in the setup with out braking the vacuum. At the bottom of the chamber an "Agilent" turbo pump of $1800 L/s$ capacity is attached to a pre-vacuum scroll pump as exhaust.
A skimmer with a diameter of $400\mu m$ is located in front of the effusive jet, sorting the gas beam not just by it size but also by its velocity vectors and allowing just those beams with direction to the further vacuum chambers. To adjust the nozzle optimally to the skimmer,it is connected to an x-y displacement unit and can be aligned from outside the vacuum chamber. To prevent the small opening of the nozzle from clogging over time, high-purity helium 6.0 (99.9999$\%$ purity) and Ne 5.0 (99.999$\%$ purity)  is used and can also be used outside of the measurements ensures a constant gas flow through the nozzle. At this extreme temperatures this prevents that any impurity in the gas bottle can  condensate, blocking the nozzle or changing the conditions of the clusters production. 

\section{Dopping chamber}

As explained in the chapter above, the Doping takes place by inelastic impacts with atoms from the gas phase, referenced as the pick-up thecnique. In this experiment we doped with both metals and noble gases, and two different methods of doping are used: Metals are heated in evaporated phase in a oven, while gas dopant yield in a Gas doping cell entered the vacuum chamber through a needle valve. on the next section we will explain the elements of the dopping cham,ber and its most important characteristics used.

The oven chamber is conected after to the Source chammber via the skimmer, it is also a 6-way CF vacuum chamber, with a turbo pump  on bottom, connected to it own pre-vacum scroll pump. On the sides the flanks allows a cold trap not used in this experiment, on the other side  the flank that permit connection to the oven and the vacuum sensor. The Skimmer is made of Niquel??, a very thin metal easy to bend, so in order to prevent stronf pressures diferrence in the chamber that can modify the skimmer, a bypass is conected between the two chamber using a stainlees steel flexible hose.  
An internal stand is welded to the front of the chamber and aligned with the skimmer. This stand supports the Chopper, the gas dopping cell and the oven.
In the front the rail the choppers is located.  It is an steel  disk with three notches uniformed located,two photocell around the bottom of the disk reads the position of the bottom notches so the upper one can be positioned right in front the skimmer. In this way, when the disk rotates the beam can pass or its block by the disk  in a controlled way.  

After the chopper, there is the Gas Dopping cell, a circular flat metal base with a self modified KF hose. The base makes the base cell have a matching patter so the hose can be easy put and remove with out losing the alignment. The stainless steel flexible hose has two $5mm$ hole (one in front and one opposite to it, some cm up the base) alligned to the skimmer so the gas bean can go thought. The hose is fix to the base and goes to the top of the chamber where it is connected to a "swaglog" niddle valve that allows to control the gas flux for doping the beam. A Pfiffer CMR375 Capacitive sensor is located after the niddle valve so a better control of the pressures, ans so of the number of dopants in the cluster can be achived. This bendable construction allows not just to remove the doping cell with out difficulty but also to fit it on the top with out depending on a fix way to located the top plank, in this way there is room for maniouver and the construction is faster.

Finally at the end of the rail lies the Oven. As shown in fig XXX the oven consist on a patterned base (similar to the gas cell) that can  move a few $mm$ on $x-y$ plane. Over it, a metal cylinder with 4 heating cartridges holds a movable crucible in the center that contains the dopant sample, this movable container is set down by a rod that comes from the top of chamber after a valve. Both, the stove and crucible have holes (a conical entrance of $40mm$ diameter and $3mm$ diameter respectively) that allows the pass of the gas beam and are aligned to the beam pad, in this way  the passing Atomic beam takes dopants via collisions with vaporized sample material.

One important advantage of this new oven design done by \textit{Dominic Schomas},  is that the dopant can be change with out braking the vacuum, it was test in this experiments and prove to be useful saving time and effort. To control the temperature of the oven a temp sensor is fix in the stove, and the resistors current is manage by a PID controller allowing a stable temperature during the experiment. The maximal temperature reached was $450^{\circ}c$,  enough to creates the gas phase for the potassium K and calcium Ca used in this experiment as shown in the table XXXX. Finally there is an  extra skimmer of diameter XXX fixed to a valve between the connection of the doping chamber and the detection chamber that helps to avoid the disperse beams or an overflow from one chamber to the other.

In addition to the doped gas nano droplets, effusive gas is also released from the dopant chamber into the detector chambers through a "swarlog niidle valve" and can be ionized and detected there. This disperse gas was pour in directly on the chamber or filtered by diffusive atoms going out of the oven and passing across the second skimmer once the choppers is close. This atomic gases where added for calibration of the detectors and background reduction allowing just one gas at a time.

\section{Detection chamber}

As mentioned, the detector chamber is connected to the Oven chamber via a valve and a skimmer. The detector chamber contains a newly developed Velocity-Map-Imaging
spectrometer on top, a time-of-flight mass spectrometer on bottom and a Lt detector on front. on this section we will give a brief presentation of  the VMi and the TOf used in this experiment, but taking spacial detaile in the new Triggering process that allows us to get the single nanoplasma explotions that we are interest on. 

\subsection{Velocity map imaging VMI}

The VMI detector used in this experiment is detail in \cite{schomas_compact_2017}, this construction basically follows the standard geometry of Eppink and Parker\cite{eppink_velocity_1997}. Its composed by three electro lenses (repeler, lens and extractor) that focuses the ions or electron on a $86,6mm$ (effective area) diameter Micro channel plates (MCP) arrange. This detector set is basically  two MCPs overposed by 90 degrees each other and a phosphoscreen (PS) layer of same diameter facing the top of the chamber to a Ca-fluoride glass of $1mm$ thick, and a CCD camera focuced on the phosphorlayer. 

\begin{table}[]
\centering
\begin{tabular}{|l|l|l|l|}
\hline
\rowcolor[HTML]{EFEFEF} 
VMI & Repeler & Extractor & Lens  \\ \hline
X1  & -2430   & -1940     & 3500  \\ \hline
X3  & -7290   & 5820      & 10500 \\ \hline
Ion & 2430    & 1940      & 0     \\ \hline
\end{tabular}
\end{table}

The voltages applied to the MCP and PS determine the brightness of the final pictures of the ions, so in general just one set of voltages were use, around $1400V$ for the MCP and $4000V$ for the Ps. The achievable energy acceptance for this stack is  $34eV$ for a the VMI setting 1 and $270 eV$ for the X3 settings. The VMNI have a resolution of $\bigtriangleup E / E\leq 4\%$ \cite{schomas_compact_2017}. The camara used in the experiment was a Basler  acA1920-155um focused on the PS.

\begin{figure}[hbtp]
\label{img:mcp cut}
\centering
\includegraphics[scale=1]{../Images/MCP cut.png}
\caption[MCP scketckht cut]{Sectional cut view of the CAD model of the spectrometer setup. On black are the are the electrodes and the white the Polyether ether ketone (PEEK) insolators. On ornange, the cooper ring ang on blue the top window facing the cCCD camera
}
\end{figure}

On fig \ref{img:mcp cut} we present a view of the model of the VMI used in this experiment, From botom to top, the structure of the electrodes consists of two repeller electrodes separated by a few millimetres with circular openings on which a fine mesh copper grid is applied, an aperture electrode as extractor, another aperture electrode which is held at ground potential and then from the extended lens electrode with the following second ground electrode. At top  the MCP-PS arrange (o black the MCP and on gray the PS) facing the center of the window instales in the top blanck of the chamber. Around thewwindow there are three conections that allows the voltages for the electrodes and the cables are carefully arrange around tthe structure to avoid discharges or even disturb the uniform electrical field.

The openings of the repeller (on bottom in bluish color) electrodes allow the use of these electrodes as well as extractor electrodes for a TOF spectrometer. In simultaneous operation of the VMI and the TOF spectrometer, the glued grids prevent mutual field effects of the two spectrometers.  The repeller and extractor are grade 2 titanium and the lens is stainless.

\section{Time of flight spectrometer}

The TOF spectrometer used was designed by Wiley and McLaren \cite{wiley_timeflight_1955}. As it names reefers, the TOF mass spectrometer relates the time that a particle on a electric field requires to reach certain distance with its mass,  when atoms and molecules are photoionized,  they pass through an electrostatic acceleration field and are registered in a detector after crossing a field-free flight path. On the basis of the flight duration the ratio m/q of a particle can be determined as:
\begin{equation}
t-t_{0}=a\sqrt{\frac{m}{q}}
\end{equation}

Where $a$ is a experimental factor depending of the flight distance, electric fields and material of the setup,  $m/q$ is the relation mass - charge, $t_{0}$ is the time ionization time (given by the laser )and $t$ is the time of flight.

The ions creation takes place between the planar repellers and extractor aperture electrodes. Behind the extractor electrode there is a further aperture electrode, which is held at zero potential and thus generates a further flight route with a defined field, Grids are glued to the openings of the electrodes to prevent the propagation of the fields through the orifices in the electrodes. The repeller electrode is set to a positive potential and the extractor to a negative potential. The resulting electric field accelerates the ions through the openings in the flight tube, on which grids are mounted on both sides to keep the drift path free of field.
Onece the coulomb explotion takes places, the ions are accelerated by the electric field of the repeller and then fly through a field-free drift path to the detector. This allows a complete mass spectrum to be recorded within a few microseconds in a single measuring step \cite{mobius_time--flight_2016}.

\begin{figure}[hbtp]

\centering
\includegraphics[scale=1]{../Images/Cup scintillator.png}
\caption[TOF cup]{Illustration of the functional principle of a Daly detector.}
\end{figure}

The flying Ions finally are detected by a  Daly detector \cite{daly_scintillation_1960}. It consists of a cup, a grid and a ring. The cup lies on negative High voltage while the ring on positive. The ions from the repeller pass trough the field free tube corrected by a drift electrosde.  Since in the direction of the hole the potential for electrons drops sharply. The ring electrode generates an E-field, which directs the ions into the cup. The ions that pass the  the grid at high speed hits on the bottom of the cup, they generate some electrons, which are transmitted through a small hole in the Cup and then in a scintillator Eljen thechnilogy(EJ-204) that flashes some photons in the process. These are detected in a Hammamatzu r-647 photomultiplier. The voltage output of the photomultiplier is controlled by a fast analog-to-digital converter, and its used at a $900V$ voltage. The cup and the drift-scintelator  tube where set at high voltages,  $-17000V$ and $-4000$ respectively.

\section{Lt detector}

The Langmuir-Taylor detector (LT detector)  chamber consists of a small CF40-6-way chamber. The detector consists essentially of an annealing filament, which is located between two planar round electrodes. The Operating Principle of an LT-Detector 
is based on surface ionisation by the tunnel effect \cite{delhuille_optimization_2002}. For the annealing filament is typically used for rhenium, platinum or tungsten, as this is the most common has a comparatively high electron work function. As a consequence, a passing neutral atoms is ionized by the heated wire, releasing an electron into the wire. The resulting ions is attracted by the negative electrodes around the wire generating a current. The ionic current generated at the electrodes is proportional to the number of ionized atoms and is measured using a picoampermeter.
The LT chamber is connected to the VMI chamber via an orifice plate and is used mainly ad a beandump also as a alingment detector. The most current generated on the chamber the most atoms arte passing thourgtthe  hole, so we can be sure that the beam pad is in the midle of the VMI repellers and extractor and the reaction takes places in the rigth area. 

\section{Camera and trigger protocol}

The idea of this master thesis was to achive individual single shots nanoplasma explosion data  in the camera (VMI) and TOF. The advantage of this correlation is the ability of threat the data in invidvidual ways, reduce the bacground in the images and achive possible properties in the coulomb explosion that are not possible when threated with averaged data.

in order to achive it, a first aproche was done doing a sofware triggering in the CCD camera software and the osciloscope for the TOF, an Acqiris Card CC103. The main idea was using LabView an external clock (a RasberryPi) was triggered by the laser, when the explosion should start, and at the same time it software trigger the camera and osciloscope programs to start the acquicition. Having all the data acquire in the same program would allow to sort the data online and reduce the storage needed to the experiment. The Labview program was tested unsusesfull for the data acquisition rate needed in ELI $100KHz$, the main problem where that using software triggering  more delay are aplied due the operating system and the comunication protocols, so even the data where acquire at the same time the delays at saving the information in the hard drive made impossible to correlate the signals.

Based on this same idea, a second aprove was used. In replacement odf a software ttrigering a ghardware triggering was used. The main idea remained, the laser triggers a dely generator that at the same time triggers the osciloscope, a R&S RTO2000 with bandwith of 600MHZ to 6GHz,  and the camara. Two facts had to been taken into account. First, our camera can´t go lower than $34\mu s$ in exposure time. Second, the timing between the camera reciving the trigger and starting the acqisition was n ot negligible as it was for thze osciloscope, as we measure, the camera took between $5-6 \mu s$ to start after the trigger was send. To solve this problem thetriggering scheme in fig XXX was used. 

\begin{table}[]
\label{tab:delaystriger}
\begin{tabular}{ll}
\multicolumn{2}{c}{List delays}                                          \\ \hline
\multicolumn{1}{|l|}{Channel} & \multicolumn{1}{l|}{Set to:}    \\ \hline
\multicolumn{1}{|l|}{A}                & \multicolumn{1}{l|}{$=T+0$}       \\ \hline
\multicolumn{1}{|l|}{B}                & \multicolumn{1}{l|}{$=T+1\mu s$}     \\ \hline
\multicolumn{1}{|l|}{C}                & \multicolumn{1}{l|}{$=B or B+6\mu s$} \\ \hline
\end{tabular}
\end{table}
A delay generator (Stanford Research Systems MD DG335) recive the laser trigger (100KHz)
channel B and C where conected to the osciloscope to chanel 1 and 2 respectivelly, and chanel A was conected to the pin 1 (trigger) on the camera. 
Lets remember that the due the minimal exposure time of the camera, we can not identify a single laser shot with it. Table \ref{tab:delaystriger} shows the deays used in the experiment, where $T$ is the original laser trigger and  A,B and C are the channels in the delay generator. In this way, it can be shown in fig, that the osciloscope can "see" each of the laser shots individually but the camera will see at least 3 shots, but fortunatelly, not each laser shot generates signal, as show in the next chapter in general just $10 to 20\%$ of the laser short ignites a plasma explosion, this mean that almos most of the pictures will have no signal, some of them can have one or more explosion, but the main of them will contain just one signal explosion in the vmi that in the data anaöisis can be correlated to its individual TOF signal. 
\begin{figure}[hbtp]
\label{fig:triggers}
\centering
\includegraphics[scale=1]{../Images/Trigger scheme.png}
\caption[Trigger Scheme]{Schem of the trigger system used on }
\end{figure}

In Fig \ref{fig:triggers}, we show the simplies ´Trigger scheme used in the experiment. The osciloscope and camera is trigered by the delayed channel B. The osciloscope is set to $50\mu s$ and the camarta to the minimal exposure time. So the camara and osciloscope sees the same trigger, the osciloscope will record at least 5 laser shots, but the camera because starts later just can see three as shown. The pictures are saved in the memory RAM of the computer so the dead time after the camara is off is mandatory to give the operative system enougth time to save the data on disk and dont full the Ram. A small improvement in this system can be done if we trigger the camara with B and the oscilloscope with C, so both apparatus can start almost at the same time and no corrections needs to be done. Each of the data set are saved with a unique label that will help to correlate the data after. Once a explosion is found in the VMI pictures, we check in its corresponding TOF that it have just one signal in all five laser shots, so we can be sure that picture correspond to a single coulomb explosion, in case more than one signal is found, this picture is discard. 





\newpage
\include{Methods}
\newpage
\include{Results}
\newpage

% !TeX spellcheck = en_GB
% !TeX spellcheck = en_US 

\chapter{Summary and Discussion}

In the proceeding chapter a detailed description of the data at similar parameters is discuss, followed by the relation of the energy distribution dependence with the number of electrons, relating it to the analytical model described in section 1.4.4.

First, we will compare the helium clusters in NIR and MIR for their pulse duration and laser intensity and subsequently compare it to the dopant elements and its role in the efficiency of the plasma formation. Additionally, a He-Ne clusters size in MIR discussion is presented to analyze the role of the cluster element at different sizes and dopants. We will pay special attention to the Xe-Ca doping experiment in He clusters to analyze the efficient of having two different dopants.  


Finally, in order to relate the data to the analytical model, the number of electron with the max energy of the single coulomb explosion were correlated in each data set to associate them to energy distribution of the uniform charge spherical model. For this, we assume that the maximal kinetic energy detected is given by the position of the electron at the outer radius of electronic cloud before the cluster coulomb exploded, and the number of electron are related to the electron charge density. The result, is a simple probability density distribution that helps to pursue an educated guess of the plasma process energies without having to do costly and time consuming simulations. Finally, we present a summary and outlook of the results of this work and the possible outcome for the suture.  

\section{Laser parameters dependence on clusters}

As shown, He cluster were ignites in NIR and MIR lasers at different laser intensities, additionally, Ne and He cluster in the same MIR field were ignited at different pulse duration. Here we present a comparison of the He at similar cluster sizes and the Ne and He cluster at the different pulses.


\begin{figure}[h!]
\hfill
\begin{subfigure}[l]{0.48\textwidth}
\caption{MIR and NIR laser intensity dependence on He clusters}
\includegraphics[width=1\textwidth]{../Images/results/Comparison_energyDistribution/Comp_Laser intensity.png} 
\end{subfigure} 
\begin{subfigure}[l]{0.48\textwidth}
\caption{Pulse dependence on Ne and He cluster in MIR laser fields}
\includegraphics[width=1\textwidth]{../Images/results/Comparison_energyDistribution/Comp_pulseduration.png} 
\end{subfigure} 

\hfill
\caption[Laser parameter comparison]{Comparison of the pulse duration and laser intensity. On the right, the pulse duration dependency for the signal rate and mean values of He and Ne clusters. On the left, the laser intensity dependence for He cluster in NIR at 800 nm wavelength and MIR at 3200 nm wavelength.  The mean number of electron is normalized to the cluster size in both cases. }
\label{fig:lasercompar}
\end{figure}

Fig \ref{fig:lasercompar} shows the comparison for the experiments with different laser intensity and pulse duration. On the left we show the signal rates and mean values for the different laser intensities using He clusters interacting with the MIR and NIR laser. On one hand, there exist a linear dependence where the intensity increase the signal rate drastically, with both lines (blues and red) having a zero signal close to 8$\cdot$10$^{13}$ W/cm$^{2}$. As expected, there exist a minimum beam intensity to achieve plasma formation because the main energy transfer from the photons to the clusters is done via ionized dopant electrons that initiate the electronic cascade and forms the coulomb explosion. If the initial energy is not enough to ionize the dopant, none electrons will interact with the laser field and the plasma will not be create. On contrast, the signal rate in the MIR can be comparable to the NIR even at lower intensities, for example, He rate for the blue line at 2$\cdot$10$^{14}$ W/cm$^{2}$ is the same as the point at 8$\cdot$10$^{14}$ W/cm$^{2}$ in the blue curve. It also can be related to the doping level and the slightly size difference in each easement but both lines have a parallel behaviour that confirm our assumption 

On the other hand, despite the normalization at the cluster size, is clear that the NIR field produce more electrons at higher energies, reaching energies one order of magnitude higher than the MIR.  This could be interpreted that the 800 nm wavelength range in the plasma resonance and the cluster is ionized completed, even for the biggest droplets, as shown in the plateau for intensities upper than 1$\cdot$10$^{15}$ W/cm$^{2}$. In contrast, assuming a complete ionization and that the detector can reach almost all electrons coming out of the explosion, in the mean number of electrons the blued curve does not have such high values as the red, what could indicate that the plasma formation in MIR laser fields is incomplete or partial. 

Fig \ref{fig:lasercompar} b. shows the pulse duration dependence of big He and Ne cluster in the MIR laser. Although in Ne a vast part of the pulse measurement is missing, we still see some relation in the points around 60 to 100 fs. On top, the red signal rate presents an expected decrease in accordance to the laser intensity plot, the longer pulses have lower intensity, in consequence the signal rate goes down. Same happens for the blue curve, that in the 60 to 100 fs  range where it behaves parallel to the He line.
In contrast, a surprising result is shown in the mean values contrasting to the intensity dependence results. It shows that at longer pulses higher energies and electron numbers are achieve, in other words, it enhance the plasma formation for bigger droplets. Although, it is a non-intuitive results, it can be explained if we take into account that for longer pulses we have more cycles in the pulse. In consequence, even the initial ionization probability is lower for the first cycles, the electrons created on them will have more time to interact with the laser field, acquiring energy and boosting the electronic cascade to end in the coulomb explosion.

\section{Cluster Size Dependence}

As shown in fig \ref{fig:elementall} the droplet size plays an important role in the nanoplasma explosion. All clusters have a commune linear behaviour linked to the cluster size, the bigger the cluster more signal can be found.  On one hand, Ne cluster shows a more stiff tendency, its signal rates goes from 4 to 12$\%$  in small droplets compared to the He$_{NIR}$ at the smallest size does not goes up the 3$\%$. Moreover, either He$_{MIR}$ and He$_{NIR}$ have a continues growth for cluster around a few hundreds of thousands. In contrast at the biggest He$_{MIR}$ cluster shows a depletion of the signal, suggesting that exist a maximal cluster size where that MIR laser is not efficient any more, contrary to the NIR where at the same size, presents a enhance of its efficiency and for huge clusters reach a maxima according to the data. 

\begin{figure}[h!]
\caption[Cluster size comparison]{ Cluster size dependence of He and Ne cluster in MIR laser fields (yellow and blue), and He cluster in NIR laser (Red)the mean values are in logarithmic scales and the signal rate in linear scale.}
\centering
\includegraphics[width=0.6\textwidth]{../Images/results/Comparison_energyDistribution/Comp_clusterSize.png}
\label{fig:elementall}
\end{figure}

On the other hand, the mean values show a similar behaviour in the MIR independent of the cluster element, lines blue and red, have a similar values, having an initial constant trend and the bigger sizes have higher counts, yet in contrast the mean number of electrons  for the smallest droplets reach counts 2 order of magnitude lower. Same happens in the mean energy where the bigger droplets show higher energies, and all lines shows a similar trend. 
The reason for this tendency is not clear, but if we assume the interaction dopant-cluster, small cluster could have a high probability of losing the ionized electrons in the beginning of the pulse, so the ignition process is less efficient. However, for the bigger clusters, the ionized electron have more atoms to interact, so the losses reduces and in consequence the bigger clusters can be more efficient to create the nanoplasma. The reason for this tendency is not clear, but if we assume the interaction dopant-cluster, small cluster could have a high probability of losing the ionized electrons in the beginning of the pulse, so the ignition process is less efficient, while for the bigger clusters, the ionized electron have more atoms to interact, so the losses are less, and in consequence the bigger clusters are more efficient to create the nanoplasma.




\section{Doping Dependence}

Fig \ref{fig:dopcompari} shows an overview of all the mean values and signal rate for the He droplets with Ar, Xe and water, and Ne doped with Xe at different doping levels. On top the signal rate shows for He$_{2.5\cdot 10^{5}}$ similar values for low and high dopant. In contrast for the Ne$_{5000}$ at extreme low levels the signal rate goes to zero.  This lack of dynamics is an expected result, as explained, the MIR laser field does not have enough energy to ionize directly the He or Ne, so a dopant with a lower IP is needed in order to start the process. Once the process is started and the electronic cloud is formed, the extra  electrons given via the dopant ionization does not change the final result. The green line, Ne-Xe, is the only graph that have important changes. It suggests that the difference on cluster size between the Ne$_{5000}$ and He$_{2.5\cdot 10{5}}$ affect the cross section to get dopants and in consequence a minimum cluster size-doping preassure relation need to be achieve to effectively pickup Xe and so the plasma formation can start. This results shows that the doping can increase the efficiency to ignite the nanoplasma, and at large doping level the mean value of electrons decrease, meaning that we are igniting small droplets. In short, is more efficient to ignite bigger droplets than small ones, because according to the data, the small ones need a larger amount of doping to be ignited while the large cluster just need a few atoms.

\begin{figure}[hbtp]
\caption[Doping level comparison]{doping level dependence of He and Ne cluster in MIR laser fields doped with water, argon, and xenon in MIR fields.}
\centering
\includegraphics[width=0.6\textwidth]{../Images/results/Comparison_energyDistribution/Comp_AllDoping.png}
\label{fig:dopcompari}
\end{figure}

The mean values also shows a similar dynamics with steady He clusters independent on the dopant or the dopant level for He, while the Ne cluster present a constant grow for low doping levels showing a plateau around 50 atoms and a depletion for higher doping. It’s clear that, for the Ne line, the increase of dopants leads to an increase on the plasma formation, yet in case the dopant is to large, can be totally destroyed due the evaporation and cluster shrinkage, similar to the reports in other experiments \cite. 

Fig \ref{fig:XEcasig}, \ref{fig:XEcaelect} and \ref{fig:Xecaenerg}, shows the signal rate and mean values interpolation for the combination of Xe and Ca in He droplets in the MIR laser. The dashed color line are cuts at a constant dopant numbers, Xe+Ca at $40,50,65,85,100,120$ and $150$ atoms. The lines were choose in order that it lays between 3 data points, so the interpolation is accurate enough. On the right of each figure the cuts are plotted depending on the Xe atoms. It means that the right side of each line represent the doping with the maximum Xe atoms at the Xe+Ca constant, once we go from right to left, we replace each Xe atom for one Ca atom until all Xe atom are replaced. Each line have a maximum of 40 points, because it was the max Ca doping measured with enough statistics. 

\begin{figure}[h!]
\hfill
\begin{subfigure}[l]{0.48\textwidth}
\caption{MIR and Nir intensity dependence on He clusters}
\includegraphics[width=1\textwidth]{../Images/results/MIR_He_XeCaDop/interpolationSignalRate.png} 
\end{subfigure} 
\begin{subfigure}[l]{0.48\textwidth}
\caption{Pulse dependence on Ne and He cluster}
\includegraphics[width=1\textwidth]{../Images/results/MIR_He_XeCaDop/interpolationSignalRatelines.png} 
\end{subfigure} 
\hfill
\caption[Xe-Ca interpolation signal rate]{On the left, signal rate interpolation for He cluster doped with Xe and Ca at different doping levels.The white points correspond to the actual measurements and the dashed line the cuts at constant number of dopant atoms. On the right, the cuts for at a constant dopant numbers, Xe+Ca at $40,50,65,85,100,120$ and $150$ atoms for the corresponding color lines.}
\label{fig:XEcasig}
\end{figure}

An important result can de derive from Fig \ref{fig:XEcasig}, as shown in the cut lines at constant total doping, we found a increment in the signal as the Ca doping replace the Xe. In all line, from right to left, the signal rate goes up when a few atoms of Ca are added despite the amount of Xe. It demonstrate a better efficiency in the doping for Xe-Ca combination compared to only Xe or only Ca. This tendency can be spot not just on the big peak in the interpolation between 70 to 120 Xe atoms and 15 to 25 Ca atoms, but also in each of the cuts, where the lines clearly display a peak, the signal rate increase faster for these extra Ca atoms added. Once a saturation maxima is reach, the extra Ca atoms added makes the signal rate again decrees but in a slower way in most of the cases. 

\begin{figure}[h!]
\hfill
\begin{subfigure}[l]{0.48\textwidth}
\caption{MIR and Nir intensity dependence on He clusters}
\includegraphics[width=1\textwidth]{../Images/results/MIR_He_XeCaDop/interpolationElectr.png} 
\end{subfigure} 
\begin{subfigure}[l]{0.48\textwidth}
\caption{Pulse dependence on Ne and He cluster}
\includegraphics[width=1\textwidth]{../Images/results/MIR_He_XeCaDop/interpolationEleclines.png} 
\end{subfigure} 

\hfill
\caption[Xe-Ca interpolation num electrons]{ On left, the mean number of electrons interpolation for He cluster doped with Xe and Ca at different doping levels. On the right, the cuts for at a constant dopant numbers, Xe+Ca at $40,50,65,85,100,120$ and $150$ atoms for the corresponding color lines.   }
\label{fig:XEcaelect}
\end{figure}

Fig \ref{fig:XEcaelect} show one clear peak at low doping where the maximum  number of electrons appears, then for higher doping the counts starts to decreases and even a small depletion is shown in the range of Xe=100 Ca=20 atoms. On the right, the cut at constant dopant is show, the color from left to right represent the same cut done in the corresponding interpolation, and it’s necessary to read it at the same way. Where the right side of each line represents just Xe doping and each number to the left mean replacing one Xe atom for one Ca atom. As show, the exist also a recurrent peak efficiency in the nanoplasma ignition, meaning that at this peak the larger cluster are ionized, similar as shown on the Pulse duration scan.


\begin{figure}[h!]
\hfill
\begin{subfigure}[l]{0.48\textwidth}
\caption{MIR and Nir intensity dependence on He clusters}
\includegraphics[width=1\textwidth]{../Images/results/MIR_He_XeCaDop/interpolationMax.png} 
\end{subfigure} 
\begin{subfigure}[l]{0.48\textwidth}
\caption{Pulse dependence on Ne and He cluster}
\includegraphics[width=1\textwidth]{../Images/results/MIR_He_XeCaDop/interpolationMaxlines.png} 
\end{subfigure} 

\hfill
\caption[Xe-Ca interpolation max energy]{On left, the mean max energy interpolation for He cluster doped with Xe and Ca at different doping levels. On the right, the cuts for at a constant dopant numbers, Xe+Ca at $40,50,65,85,100,120$ and $150$ atoms for the corresponding color lines. }
\label{fig:Xecaenerg}
\end{figure}

Equally important. In the E$_{max}$ interpolation, a clear peak is also shown at Xe= 50 and Ca=7 atom,  This can be found in interpolation cuts, where the highly doped lines have a more constant performance and no real change is shown, but once we get close to the optimum doping (Yellow line), the replacement of Xe atom for Ca atoms becomes drastic effective, with a peak  founded close to 0.2 eV.  This measurement can be compare directly to the water doping scan where we show that once the ignition of the cluster starts, adding more water atoms does not creates changes in the process. Here, Xe doping results in a similar behavior, the points with just Xe atoms (Right edge of each colored line) have a relative constant mean energy, but once we introduce the Ca, dynamics start to appears, given a peak close to the Mean electrons interpolation. For the heavily doped clusters, on the blue, orange and purple cuts in the right, we see that the peaks are not present any more due the cluster destruction when to many atoms are added.



\section{Electron-Energy Distribution}

Once the number of electron and max energy was extracted from each VMI picture, an energy distribution depending on the electrons detected can be plot to fitted to the uniform charge spherical model explained in chapter 1. The next figures will display the binned correlation for the max kinetic energy and number of electrons with their correspondent fit based on $E_{max}=B\cdot n^{2/3}$ Eq. 1.30, where $n$ is the number of electrons  and $B$ is the B-factor related to the electron density in the electronic cloud. The error bars are the standard derivation for the binned section and in case there is just one point in the binned region the error was assigned to the resolution of the experiment.

Fig \ref{fig:caxeinter} presents the data for He droplets in NIR laser fields for the cluster size dependence and the laser intensity measurements. Fig a. shows the binned energy distribution for the different droplets sizes. The red and purple line is a fit based on Eq. 1.30 where B is the B-Factor and $n$ is the number of electrons, the $2/3$ factor is fix to the fitting. The red and purple line correspond to the biggest and smaller droplets respectively, and all other fit lines showed the same tendency fit quite well to the data points. For the B-factor at Ne$_{4.6\cdot10^{5}}$ the corresponding electronic cloud density is around $\rho =2.2 [1/\mu m^{3}]$, what leads to an electronic cloud radius of  4.3 $\mu$m.
In Fig b. the same process is done for each of the data sets at different laser intensities. All fits have the same tendency regardless the laser intensities, changing slightly on the B-factor. This in an important outcome meaning that the data fits quite well to our simple spherical electronic cloud model. As the B-factor is directly related to the density of the electronic cloud using Eq. 1.3 is possible to delimit the radii of the electronic sphere. In other words, according to this model, if we know the total number of electrons it’s possible to calculate the maximal energy that we can detect. For the B-factor at 1.4$\cdot10^15$ W/cm$^{2}$ the corresponding electronic cloud density is around $\rho =0.02 [1/\mu m^{3}]$, what leads to an electronic cloud radius of  136 $\mu$m. The Inset on the top-left cornet shows the electron density for each of the laser intensities and cluster sizes, showing a strong dependence quit the cluster size a s expected but a similarity with the signal rate intensity dependence with a plateau in the higher laser intensities.


 \begin{figure}[h!]
\centering
\begin{subfigure}[t]{0.65\textwidth}
\caption{He cluster size dependence energy distribution in NIR laser pulses}
\includegraphics[width=1\textwidth]{../Images/results/NI_He_Dropletsize/vinned.png} 
\end{subfigure} 
\hfill

\centering
\begin{subfigure}[t]{0.65\textwidth}
\caption{Helaser intensity Dependence Energy distribution in NIR laser pulses}
\includegraphics[width=1\textwidth]{../Images/results/NIR_He_intensityscan/binned.png} 
\end{subfigure} 
\hfill
\caption[Energy-Number of electrons relation. NIR Helium Droplets]{Energy distribution related to the number of electron for He clusters in NIR laser fields at different Cluster size fig a. and laser intensity Fig b.}
\label{fig:NIRcaxeinter}
\end{figure}


Fig \ref{fig:Heenrgd} presents the data for He droplets in MIR laser fields for the cluster size dependence and the Pulse Duration measurements. Fig a. presents the binned maximal kinetic energy distribution for different droplet size, the color points shows the different cluster sizes with the standard dispersion as error bars. The continue lines are fit lines with $c$ exponent fixed to $2/3$. It is clear that the continue lines deviates from the data, meaning that the spherical model cannot be applied for these measurements. Moreover, the dashed line is a fit based on the same equation but with the exponent $c$ free. As seen, this lines have a better agreements with the data, even though the analytical model cannot be applied in this case, there exist a clear correlation of the energy and number of electrons that a refinement of the model could solve in a future. This new exponent is persistent in all the fit for the MIR experiments in helium.  Fig B show another example where the spherical model (continues lines) fails and a better fit with the $c$ match the data (dash line)

There is no theoretical background that can predict this behaviour, one of the possible explanation, taking into account that the exponent is closely linked to the geometry of the cloud, is that the big droplets are not perfectly spherical, for example, because He is liquid, the droplets could turn in an ellipsoidal shapes. . A second reason can be because of the partial coulomb explosion described in the previous section. As describe, not all the cluster is generating a plasma and just part of it generates a nanoplasma, so the final electronic cloud will present abnormalities making its density and shape non uniform. Although our analytical model cannot describe precisely the energy distribution it is clear that the number of electron and max energy have a close relation and even the model does not fit, it can be possible to introduce some correction in the future to have a better model

 \begin{figure}[h!]
\hfill
\begin{subfigure}[l]{0.48\textwidth}
\caption{He cluster size dependence energy distribution in MIR laser pulses}
\includegraphics[width=1\textwidth]{../Images/results/Mir_He_Dropletsize/binned.png} 
\end{subfigure} 
\begin{subfigure}[l]{0.48\textwidth}
\caption{He pulse duration Dependence Energy distribution in MIR laser pulses}
\includegraphics[width=1\textwidth]{../Images/results/MIR_He_pulsescan/raw/binned.png} 
\end{subfigure} 
\hfill
\caption[Energy-Number of electrons relation. MIR helium droplets]{Energy distribution related to the number of electron for He clusters in MIR laser fields at different Fig. a. and cluster sizes and pulse duration Fig. b.}
\label{fig:Heenrgd}
\end{figure}

Fig \ref{fig:Neenrgd} presents the data for Ne droplets in MIR laser fields for the cluster size dependence and the Xe dependence measurements, with $c$ and $B$ factor set free. In Fig a. The fit function are done for the shorts pulse with the c exponent at $2/3$ and as a free parameter. Once again the model does not fit but a clear energy distribution is present independent for the pulse duration. The dashed line follows $B=0.03$ and $c=0.23$ and is shared for the other data sets. 
Figure b. shows the Energy distribution with a similar trend, the data point and signal presents the same energy distribution as shown in the pulse scan. The fit function was done in the same way. The model fit shows and proximate better agreement to the first data point at lower energies but diverge rapidly to the bigger ones. The dash fit on the contrary show that the energy distribution follows a similar trend independent to the cluster size, but our spherical model cannot be applied.

 \begin{figure}[h!]
\hfill
\begin{subfigure}[l]{0.48\textwidth}
\caption{Ne Pulse length Dependence Energy distribution}
\includegraphics[width=1\textwidth]{../Images/results/MIR_Ne_pulseduration/binned2.png} 
\end{subfigure} 
\begin{subfigure}[l]{0.48\textwidth}
\caption{Ne Cluster size Dependence Energy distribution}
\includegraphics[width=1\textwidth]{../Images/results/MIR_Ne_DropletSize/binned.png} 
\end{subfigure} 
\hfill
\caption[Energy-Number of electrons relation. Neon Droplets]{Energy distribution related to the number of electron for Ne clusters in MIR laser fields at different pulse duration fig a. and cluster sizes Fig b.}
\label{fig:Neenrgd}
\end{figure}


\newpage\newpage

\listoffigures
\listoftables


\bibliography{thesis}
\bibliographystyle{abbrvdin}



\chapter*{Danksagung}

An dieser Stelle Danke 
\newpage
\section*{Erklärung}

Ort, Datum ............................... \ \ \ \ \ \ \ \ \ \ \ \ \ \ \ \ \ \ \ \ \ \ \ 
Unterschrift ................................
 
\end{document}
	