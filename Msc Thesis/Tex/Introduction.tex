% !TeX spellcheck = en_GB
% !TeX spellcheck = en_US

\section{Introduction}

Physicists always have been fascinated to explain and resolve dynamic processes in short leaps of time of times and to characterize in a time scale fast processes with highly physical interest. Describing these systems requires to acquire data in brief time windows, for example, a film is only a consecutive sequence of photographs that recreate a large time laps in a smaller time scale. For atomic physics, we are talking about a micro-cosmos that can go down to attoseconds. Experiments in attoscience have given up to Nobel prizes, for example  Ahmed Zewail  in 1999, who demonstrated that femtosecond pulses allow a temporally resolved observation of ion movements in molecules \cite{zewail_femtochemistry:_2000}.
 
The duration of the dynamics of a system is related directly to its energy and its quantum states, in other words, to its size. For example, to explore ion dynamics in molecules, times can fluctuate between fentomoseconds to nanoseconds \cite{gruner_femtosekundenspektroskopie_2013} and energies up to several electron volt eV. Furthermore, for smallest system as the relaxation of an inner shell vacancy \cite{drescher_time-resolved_2002} or the process of tunneling ionization \cite{uiberacker_attosecond_2007} occur in attosecond regime, with energies that varies typically on the milielectronvolt scale (meV) for its energy levels. And further, for processes as nuclear fission and quasifission are predicted to unfold even faster, typically on a zeptosecond time scale\cite{ray_quasifision_2015}.

To study ultra-short-dynamics, physicist have been challenged to create detector systems based on the interaction light-matter. Along the last decades, experimental schemes have been devised developing new light sources, as for example the laser, to interact with the microcosms. Nowadays, laser pulses can reach up to extreme non-linear optical processes, producing single isolated  pulse of ultraviolet(UV) waves as short as $67$ as \cite{zhao_tailoring_2012}.  Such fast pulses open up the possibility of time resolved measurements for short processes like nuclear dynamics, the generation of high energetic electrons and ions in the KeV-regime \cite{fennel_laser-driven_2010}, as well as intensive XUV and attosecond pulse experiments \cite{stebbings_generation_2011}. Not just the pulse duration has been improved within the years, but also the energies that the pulse can deliver, laser pulses with intensities up to 10$^{21}$ W/cm$^{2}$  are available  commercially \cite{mourou_optics_2006}.
 
Femtosecond and attosecond lasers pulses are a milestone in the control and ignition of atomic processes. These advances have enabled the development of new research areas, as well as the investigation of ultrafast dynamics of highly excited matter on a nanometer scale. In particular, many studies have been published in recent years on atomic and molecular clusters that investigate their interaction with NIR, UV or XUV pulses\cite{stebbings_generation_2011}. These are motivated by a broad spectrum of possible applications such as the generation of energetic electrons and ions in the keV regime \cite{fennel_laser-driven_2010}.

In this thesis, we focus our efforts on the ionization processes by Near-Infrared (NIR) and Mid Infrared (MIR) femtosecond pulse in doped noble gases clusters. The interaction of the dopant with the laser field results in an energy transfer to the cluster that starts an ionization process, known as a nanoplasma. The oscillation of the electrons driven by the laser field inside the clusters creates an energy transfer that ionize the cluster atoms \cite{fennel_laser-driven_2010}. This process, caused predominantly by electron impact ionization, makes an avalanche-like ionization of the atoms in the cluster, leading to a heating of the plasma and, as a result, a hydrodynamic expansion and Coulomb explosion. Here, we studied the electrons and ion's resulting from the coulomb explosion using a velocity map imaging and a Time of flight spectrometer set in parallel to acquire the data and reconstruct the initial energy  configuration of the plasma.


In the First chapter, we will present a brief introduction to the Helium and Neon cluster generation, a short explanation of the ionization process and the basic background for the plasma formation and coulomb explosion process. In the second chapter, we show a detailed explanation of the setup used, explaining from the creation of the Helium droplets process to its detection, describing as well the doping and ignition process. Finally, a detailed explanation of the correlation method for the VMI-TOF measurements is done, showing the set-up of the data acquisition and its advantages.  In the third chapter, we discuss the acquisition method and the calibration methods used in the experiment as well as a brief explanation of the analytical model for the coulomb explosion, followed by the description of the data analysis method. Finally on the four chapter, we present the results and its analysis followed by a summary and outcome of the experiments as well as the inputs for future works to improve the results.
