% !TeX spellcheck = en_GB
% !TeX spellcheck = en_US

\section{Introduction}


Physicists always have been fascinated to explain and solve dynamic processes in short amounts of times and to characterize fast processes with highly precision. For atomic physics, dynamic processes happens down to attosecond (10$^{-18}$ s) time scales. Experiments in attoscience have given important results culminating in  Nobel prizes, for example the Nobel prize in chemistry awarded to Ahmed Zewail  in 1999, who demonstrated that femtosecond  (10$^{-15}$) pulses allow a temporally resolved observation of ion movements in molecules \cite{zewail_femtochemistry:_2000}.

The duration of the system dynamics is related directly to its energy and its quantum states. For example, to explore ion dynamics in molecules, times can fluctuate between femtoseconds to nanoseconds (10$^{-9}$) \cite{gruner_femtosekundenspektroskopie_2013} and energies up to several electron volt (eV). Furthermore, electronic dynamics as the relaxation of an inner shell vacancy \cite{drescher_time-resolved_2002} or the process of tunneling ionization \cite{uiberacker_attosecond_2007} occur in the attosecond (as) regime, with energies around the millielectronvolt scale (meV). 

To study ultra-short-dynamics, physicist have been pushed to create experiments where the light-matter interaction plays an important role. Along the last decades, experimental schemes have been conceived using new light sources, such as femtosecond (fs) pulsed lasers, to interact with atoms or clusters. Nowadays, lasers can reach up to extreme non-linear optical processes, producing single isolated extreme ultraviolet (XUV) pulses as short as $67$ as \cite{zhao_tailoring_2012}. Such fast pulses create the opportunity of time resolved measurements of short processes like electronic dynamics, the generation of high energetic electrons or ions in the keV up to MeV regime \cite{fennel_laser-driven_2010}, XUV and attosecond pulse experiments \cite{stebbings_generation_2011}. Along with the reduction of the pulse duration, the peak intensity has increased too during the las decade, laser pulses with intensities up to 10$^{21}$ W/cm$^{2}$  are available commercially \cite{mourou_optics_2006}.
 
Femtosecond and attosecond laser`s pulses are a milestone in the control and ignition of atomic processes. These advances have enabled the development of new research areas as the nanoplasma dynamics. Many studies have been published in recent years on atomic and molecular clusters that investigate the cluster interaction with Near-Infrared (NIR), UV or XUV pulses\cite{stebbings_generation_2011}. The studies are motivated by a broad spectrum of possible applications such as the generation of electrons and ions in the KeV up to MeV regime \cite{fennel_laser-driven_2010}, and high energetic photons \cite{schutte_strong-field_2016}
.  
The Mid-Infrared (MIR) regime is particularly interesting, due the ponderomotive energy (the average energy gain by a particle in a laser field) scales with the wavelength squared (U$_{P}\sim\lambda^{2}$). A very prominent example for the useage of longer wavelengths is the cutoff law for high order harmonic generation in rare gas atoms, where harmonics with energies of $I_P + 3 \cdot U_P$ can be observed, where $I_P$ is the ionization energy of the rare gas atoms \cite{krause_high-order_1992}. Some studies have observed a strong increase in the kinetic energy from around 300 eV up the KeV regime for argon, krypton and xenon clusters for $1800$ nm laserpulses compared to 800 nm pulses, both at similar intensities around $1\cdot10^15 $W/cm$^2$ peak intensities.\cite{schutte_strong-field_2016}

The possibility to generate controlled heterogeneous clusters consisting of different species allows to investigate the nanoplasma dynamics depending its composition and the laser parameters that act on it. This is of scientific interest not only because of the potential applications as particle source, but also because the droplets can act as a matrix for chemical reaction for embedded large molecules or biophysics experiments such as viruses or proteins imaging\cite{stienkemeier_spectroscopy_2006}. To assets the especial properties of He droplets we have produce systematic comparative experiments using Ne droplet in Mid-Infrared femtosecond pulses, comparing the behaviour under different, laser parameters, dopants and cluster sizes.

This work is focused on the ionization processes by NIR and MIR femtosecond pulse in doped noble gas clusters. The interaction of the dopant with the laser field induces an energy transfer to the cluster that starts ionized process resulting in highly ionized state of the cluster, known as a nanoplasma. The plasma formation starts when the ionized dopant electrons are driven by the laser field inside the clusters creating an energy transfer that ionize the constituent atoms \cite{fennel_laser-driven_2010}. This process, caused predominantly by electron impact ionization, makes an avalanche-like ionization in the cluster, leading to heating of the plasma and a hydrodynamic expansion or Coulomb explosion. In this study we analyze the electrons and ion resulting from the Coulomb explosion using a velocity map imaging (VMI) and a time of flight (TOF) spectrometer to acquire the data and reconstruct the final energy configuration of the plasma.

In the first chapter, we present a brief introduction to the helium and neon cluster generation, a short explanation of the ionization process, the basic background for the plasma formation and a brief description of the analytical model for the Coulomb explosion. In the second chapter, we show a detailed description of the setup used. We describe the creation, doping and ignition process for He and Ne clusters. A detailed explanation of the correlation method for the VMI-TOF measurements is done, showing the setup of the data acquisition and its advantages. In the third chapter, we discuss the acquisition method, the calibration methods and the data analysis method used in the experiment. Chapter four shows the results and analysis of the electron-VMI measurements of He and Ne droplets driven by NIR and MIR femtosecond laser pulses. Finally, we present a comparison and summary of the different experiments with the outcome and inputs for future work.




