\section{Introduction}

Physicists have always wonder to explain and resolve dynamic processes in short times, to characterize in a time  evolution scale fast processes indistinguishable to us. Describe these systems requires to acquire data in brief time windows, for example, a film is only a consecutive sequence of  photographs that recreate a large time laps in a smaller time scale. For  atomic physics, we are talking about a micro-cosmos that varies from microseconds to  attoseconds, i.e in atoms  where time scales can go down to $10^{-9} s$.

The dynamics time window of a system is related to its quantum dynamics, in other words, to its size. For dynamics happening in a molecule or a many body system interaction,  times can oscillate between  microseconds to nanoseconds. from an energetic point of view, systems on the milielectronvolt scale $(meV)$ for its energy levels, implies that molecular vibrations occurs on a time scale of tens to hundreds of femtoseconds. In smaller complex, motion of individual electrons in semiconductor nano structures, molecular orbitals, and the inner shells of atoms occurs on progressively shorter intervals of time, ranging from tens of femtoseconds to less than an attosecond. Motion within nuclei is predicted to unfold even faster, typically on a zeptosecond time scale.

To achive this higth resolution in space-time physicist have challenged to create systems with a well controlled spatial and temporal gradient. Fortunately nowadays, laser pulses can research up to extreme non-linear optical processes, produccing single aisolated  pulses of ultra violet(UV) waves as short as $67 as$ \cite{zhao_tailoring_2012}.  Such fast pulses open up the possibility of time resolved measurements fort short processes like electron dynamics.  However, to do this, experimental schemes must be devised that allow these new light sources to be used to perform measurements on the microcosmos. In particular, in the last few years,  many studies at atom- and molecule-clusters had been published, from mid-infre red (NIR) interaction to UV or XUV pulses, that not just lead to a broad spectra to study, but also to a large range of possible applications such as the generation of  energetic electrons and ions in the keV-regime \cite{fennel_laser-driven_2010}, as well as intensive XUV and attosecond pulses \cite{stebbings_generation_2011}. Laser pulses with peak intensities up to $10^{21}$ $W/cm^{2}$  are available nowadays \cite{mikaberidze_atomic_1981} commercially so the difficulty and expensive of the experiments source also are easy.

Femtosecond lasers are a milestone in order to control and ignite atomic processes in controlled standard,but  acquire the information is also a huge challenge. Now a days, several techniques are available  depending the nature of the process. For this particular  work we are interested in two  techniques, Velocity map image (VMI) and Time of flight (TOF).
Since its invention, this techniques  has become two of the most commune and important measurement techniques in high energies physics....
%%Mising more info of VMI and TOF)
But detecting a signal is just one part of the job, the new laser advances like the  generation of coherent high-intensity laser pulses with intensities up to $10^{22} W/cm^{2}$  allow multiphoton ionization that allows to get time resolved measurements. These advances have enabled the development of new research areas, as well as the investigation of ultrafast dynamics in highly excited matter to nanometer size.

In this thesis, we focus our efforts on the ionization processes by Mid Infrared (MIR) femtosecond pulses in doped noble gases clusters. The interaction of the dopant with the laser field result in a energy transfer to the droplet that ignite a ionization process, known as a nanoplasma. This resonant interaction of the laser field with a collective oscillation of the electrons in the plasma is driven by the laser field \cite{fennel_laser-driven_2010}. This process, caused predominantly by electron impact ionization, makes an avalanche-like ionization of the atoms in the cluster, leading to a heating of the plasma and, as a result, to hydrodynamic expansion and Coulomb explosion. To the analysis of this process we studied the electrons as well as the ion's resulting in the coulomb explosion. A velocity map imaging and a Time of flight technique are set up in parallel to acquire the data and reconstruct the initial energies and configuration of the plasma in study. 
In the First chapter we will present a brief introduction to the Droplet He generation, a short plasma interactions as a basic background of coulomb ionization in order to understand the physical meaning.
In the second chapter a more detailed explanation of the set-up used. Showing from the creation of the He droplets process to its detection , going thorough the doping, and ignition process. Finally, a detailed explanation on the correlation method for the VMI-TOF measurements is done, showing the set-up of the data acquisition and its advantages.

In the third chapter we show the correlated data and its analysis, introducing the calibration methods, data processing and results. Finally, in the last chapter we present the summary and outcome of the experiments as well as the inputs for future works to improve the results.

