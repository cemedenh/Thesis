\section{Introduction}

Physicists have always wonder to explain and resolve dynamic processes in short scale times, so initial conditions of processes can be  describe in a time  evolution scale. Describe any system like this requires to acquire data in shorter windows of time, for example a film is only a consecutive sequence of  photographs that recreate a large time laps in a smaller time scale pics. For  atomic physics, we are talking about a micro-cosmos that varies from microseconds, i.e several bodies dynamics, to  attoseconds  for atoms,where time scales can go down to $10^{-9}$ $s$, requiring to create measurement methods capable to record in shorter time, while the experiment have to be done in a controllable way to ensures its reproductivility, as any scientific method.

The time window of dynamics of a sytem is related to  quantum dynamics, in a simple view also to its size. For dynamics happening in a molecule or a many body system interaction, the time window can oscillate between  microseconds to fentoseconds, although for millielectronvolt-scale $(meV)$ energy spacing of vibrational energy levels implies that molecular vibrations occur on a time scale of tens to hundreds of femtoseconds. The motion of individual electrons in semiconductor nanostructures, molecular orbitals, and the inner shells of atoms occurs on progressively shorter intervals of time ranging from tens of femtoseconds to less than an attosecond. Motion within nuclei is predicted to unfold even faster, typically on a zeptosecond time scale.

To achive this higth resolution in space and  time physicist have challenged to create systems with a well controlled spatial and temporal gradient. Fortunately nowadays, laser pulses can research up to extreme non-linear optical processes, produccing single aisolated  pulses of ultra violet(UV) waves as short as 67 $as$ \cite{zhao_tailoring_2012}.  Such fast pulses open up the possibility of time resolved measurements fort short processes like electron dynamics.  However, to do this, experimental schemes must be devised that allow these new light sources to be used to perform measurements on the microcosmos. In particular, in the last few years,  many studies at atom- and molecule-clusters had been published, From mid-infre red (NIR) interaction to UV or XUV pulses, that not just lead to a broad spectra to study but also to a large range of possible applications such as the generation of  energetic electrons and ions in the keV-regime \cite{fennel_laser-driven_2010}, as well as intensive XUV and attosecond pulses \cite{stebbings_generation_2011}. Laser pulses with peak intensitiesof up to $10^{21}$ $W/cm^{2}$  are available nowadays \cite{mikaberidze_atomic_1981} comercially so the difficulty and expensive of the experiments source also are easy.

But Having this is never enough, Lasers is just one huge step in order to control and ignite atomic processe in controlled standard. The other step needes is to acquirte the information we want from this proceses. For this porpouse several thecniques are available  depending the dnature of the process. For this particular  work we are interested in two particular thecniques, Velocity map image (VMI) and Time of flight (TOF).
Since its invention, this two thecniques has become two of the most commun and important measurement techniques in high energies physics. But detecting a isgnal is just one part of the job, the new laser advances like the  generation of coherent high-intensity laser pulses with intensities up to $10^{22} W/cm^{2}$  allow multiphoton ionization that allows to get time resolved measurements. These advances have enabled the development of new research areas, as well as the investigation of ultrafast dynamics in highly excited matter to nanometer size.

In this thesis we focus our efforts  on the ionization process by NIR femtosecond pulses in doped clusters from
$He$. The interaction of the dopped He droplet with the Laser field result in a energy transfer to the droplet  that ignite a ionization process, known as a nanoplasma. This resonant interaction of the laser field with a collective oscillation of the electrons in the plasma is driven by the laser field \cite{fennel_laser-driven_2010}. This process, caused predominantly by electron impact ionization, makes an avalanche-like ionization of the atoms in the cluster, leading to a heating of the plasma and, as a result, to hydrodynamic expansion and Coulomb explosion of it. To the analisis of this proceses two tecniques were used to study the electrons as well as the iionr resulting in the coulomb explotion. A velocity map imaging and a Time of flogth technique are set up in parallel to acquire the data and reconstruct the initial energies and configuration of the plasma in study. 
In the First chapter we will present a brief introduction to short plasma interactions an a basic backgourn of coulumb ionization in order to understant the physical meaning of the reaction.
in the secont chapter a more detailed explanetion of the set up used is donde. showintg from the ceation of the He droplets proces to the detection process, going througth the dopping, and ignition process
For the thirtd chapter a detailed explanation on the correlation met´hod for the VMI-TOF meassuments is done, and showing the proces of the data acqiuisitzion and its advantages
In the fourth chapter we present the correlkated data and its analisis. finally the last chapter we present the conclusion of the experiment itself also as the data analisis and future works will needed to improve this proces as well.


I. INTRODUCTION
Ion imaging techniques in the field of molecular reaction dynamics photofragment  imaging,  photoelectron  imaging,reaction   product   imaging have   proven   to   be   of   high value.
1–3 Especially, the capability of probing the full three-dimensional velocity distribution of scattered particles under study in a single image has contributed to the importance of this method, which has the multiplexing advantage of detecting  particles ions  or  electrons of  all  velocities  the  term velocity refers to the vector quantity whereas speed denotes the scalar! simultaneously. Because the detection of particles often  involves  multi-photon  ionization MPI schemes  the ion imaging technique is widely applicable, and is shown to compare  well  with  established  one-dimensional 1D time- of-flight TOF 4–6 and  Doppler  methods.7,8
As  in  conventional  TOF/MPI  techniques,  the  images  can  often  be  obtained  in  a  mass  and  internal state  selective  way  and,  in addition, they can provide information on orientational and alignment effects. 10
In  contrast  to  the  conventional  time-of-flight  method, where kinetic energy release information is contained in the temporal structure in the arrival period of electrons or ions of a specific mass, the ion imaging technique extracts all information kinetic  energy  and  angular  distributions from  the spatial appearance of the two-dimensional~ 2D! image. From an image the full three-dimensional 3D information can be reconstructed  by  means  of  an  Abel  inversion or  back projection method.11
This  also  implies  that  the  kinetic  energy resolution attainable is ultimately limited by the quality of  spatial  mapping  by  the  detection  system.  In  some  cases imaging appears less favorable compared to TOF methods in this respect.
In order to exploit the imaging method to its full poten-
tial  one  needs  to  explore  methods  for  improvement  of  the
spatial  quality  of  the  2D  image.  Since  the  mapping  of  3D
distributions of charged particles onto the 2D detector is par-
ticularly  dependent  on  the  electrode  configuration  used  to
form the extracting electric field, this study is concerned with
the  comparison  of  conventional  grid  electrodes  versus  the
application  of  a  simple  three-plate  electrostatic  lens  with
open electrodes, which can be classified as an ‘‘asymmetric
immersion lens.’’ It shall be pointed out how this lens avoids
distortions  commonly  present  in  imaging  with  grids,  along
with  having  additional  appealing  features.  It  turns  out  that
the ion lens can be operated such that particles with the same
initial velocity vector are mapped on the same point on the
detector,  irrespective  of  their  initial  distance  from  the  ion
lens axis. A more accurate description for the imaging tech-
nique using electrostatic lenses is therefore
velocity map im-
aging
